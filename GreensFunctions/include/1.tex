\section{Introduction}
The first part of this text is primarily concerned with solutions to differential equations of the form
\begin{equation}
    Lu=\phi
\end{equation}
over an interval \(a \leq x \leq b\) and subject to certain boundary conditions, where \(L\) is an \(n\)th order linear ordinary differential operator, and \(u\) and \(\phi\) are functions of the independent variable. For \(L\) to be linear, it must satisfy the condition
\begin{equation}
	L(\alpha v + \beta w) = \alpha Lv + \beta Lw
\end{equation}
for arbitrary functions \(v\) and \(w\), with \(\alpha\) and \(\beta\) being constant. We claim without proof that for this condition to be met, \(L\) must be of the form
\begin{equation} 
	L = a_0(x) \frac{d^n}{dx^n} + a_1(x) \frac{d^{n-1}}{dx^{n-1}} + \cdots + a_n(x).
\end{equation}
Since \(L\) is of order \(n\), there will be \(n\) boundary conditions of the general form 
\begin{equation}
	B_j (u) = c_j;\quad j=1,2,\dots,n
\end{equation}
where the \(B_j\)'s are prescribed functionals and \(c_j\)'s are prescribed constants. We will only consider \(B_j\)'s that are linear combinations of \(u\) and its derivatives through order \(n-1\) and evaluated at the endpoints, a and b. 

Here, functional refers to a transformation with a set of functions as its domain and a set of numbers as its range. To illustrate what we mean, consider the functional 
\begin{equation}
	\mathcal{F}(u) = \int_{0}^{1} u^2(x)dx.
\end{equation}
The domain of this functional might be the set of functions defined over the interval \((0,1)\) and for which the integral of \(u^2\) from 0 to 1 exists, and the range is \((0, \infty)\).

For \(B_j\) to be linear, it must satisfy the condition
\begin{equation}
	B_j(\alpha v + \beta w) = \alpha B_j (v) + \beta B_j(w)
\end{equation}