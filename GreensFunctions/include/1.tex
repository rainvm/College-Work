\section{Introduction}
The first part of this text is primarily concerned with solutions to differential equations of the form
\begin{equation}
    Lu=\phi
\end{equation}
over an interval \(a \leq x \leq b\) and subject to boundary conditions \( \{ B_1, \dots ,B_n \} \), where \(L\) is an nth order linear ordinary differential operator.  For L to be linear it must satisfy the condition
\begin{equation}
	L(\alpha v + \beta w) = \alpha Lv + \beta Lw
\end{equation}
for arbitrary functions \(v\) and \(w\), with \(\alpha\) and \(\beta\) being constant. For this condition to be met \(L\) must be of the form
\begin{equation} 
	L = a_0(x) \frac{d^n}{dx^n} + a_1(x) \frac{d^{n-1}}{dx^{n-1}} + \cdots + a_n(x)
\end{equation}
The boundary conditions are linear functionals of the form 
\begin{equation}
	B_j (u) = c_j;~~~~ j=1,2,\dots,n
\end{equation}
where \(c_j\) is an arbitrary constant. 

Here, functional refers to a transformation that has a set of functions as its domain and a set of numbers. As an example
\begin{equation}
	B (u) = u(0) = 0
\end{equation}
is a simple boundary condition for a 1st order differential operator. Specifically, our \(B_j\)'s will be limited to linear combinations of u and its derivatives up to order \(n-1\). These boundary conditions have the same linearity constraints as the differential operator L. 

