\begin{example}
    We would like to show that \(e^x\d(x)=\d(x)\). To begin with, we integrate \(e^x\d(x)f(x)\) and would like to show that this is equal to \(f(0)\) to satisfy the generalized function definition of the delta function.
    \begin{equation}
        \begin{split}
            \intR e^x\d(x)f(x)dx &= \intR \d(x)\underbrace{e^xf(x)}_{g(x)}dx\\
            &=\intR \d(x)g(x)dx\\
            &=g(0)\\
            &=e^0f(0)\\
            &=f(0)\\
        \end{split}
    \end{equation}
    Therefore
    \begin{equation}
        \intR e^x\d(x)f(x)dx = \intR \d(x)f(x)dx
    \end{equation}
\end{example}
\begin{example}
    We would like to show that \(\frac{d^4}{dx^4}|x|^4 = 12\d(x)\). Note that \(|x|=2xH(x)-x\). We begin by manipulating \(|x|^3\) into a more convenient form.
    \begin{equation}
        \begin{split}
            |x|^3 &= (2xH(x)-x)^3\\
            &=x^3(2H(x)-1)^3\\
        \end{split}
    \end{equation}
    Next, it is useful to simplify.
    \begin{equation}
        \begin{split}
            (2H(x)-1)^3 &= (2H(x)-1)(2H(x)-1)^2\\
            &=(4H^2(x)-4H(x)+1)(2H(x)-1)\\
            &=(4H(x)-4H(x)+1)(2H(x)-1)\\
            &=2H(x)-1
        \end{split}
    \end{equation}
    Finally, we take successive derivatives.
    \begin{equation}
        \begin{split}
            \frac{d}{dx}(x^3(2H(x)-1)) &= 3x^2(2H(x)-1) +2x^3\d(x)\\
            &=3x^2(2H(x)-1)\\
            \frac{d}{dx}(3x^2(2H(x)-1)) &= 6x(2H(x)-1) + 6x^2\d(x)\\
            &=6x(2H(x)-1)\\
            \frac{d}{dx}(6x(2H(x)-1)) &= 12(H(x)-1) + 12x\d(x)\\
            &= 12(H(x)-1)\\
            \frac{d}{dx}(12(H(x)-1)) &= 12\d(x)
        \end{split}
    \end{equation}
\end{example}