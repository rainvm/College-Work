\section{The Delta Function}
The Dirac delta function is integral to the method of green's functions. It is typically represented by the symbol \(\delta\) and has the properties
\begin{equation}
    \delta(x) = 0~~~ \text{for } t\neq0
\end{equation}
and
\begin{equation}
    \int_{-\infty}^{\infty} \delta(x) dx = 1
\end{equation}
but its most important property is 
\begin{equation}
    \int_a^b f(x)\delta(x-t) dt= f(t), ~~~ a\leq t \leq b
\end{equation}
If, in the previous integral, \(t\) is not between \(a\) and \(b\), then the integral is equal to 0. 

To attempt to find an intuition for the delta function, we can make use of the idea of "point actions", or actions which are highly localized in space or time. As an example, consider a sharp hammer blow, with force \(f(t)\). We do not necessarily know the shape of the force function, but 

While commonly referred to as a function, \(\delta\) is more appropriately defined as a generalized function or a distribution. 