\section{The Adjoint Operator}
To determine the Green's function for a particular differential equation and its boundary conditions, begin by finding the adjoint operator, denoted \(\mathcal{L*}\). The adjoint operator consists of the formal adjoint, \(L^*\), and the boundary conditions associated with the Green's function. To determine these, first form the product, \(vLu\), and integrate it over the interval of interest. By repeated integration by parts, we can express the integral in the form
\begin{equation} \label{eq:adjoint}
	\intl vLu dx = [\cdots]\biggr\rvert_\mathrm{a}^\mathrm{b} + \intl uL^*v dx,
\end{equation}
where \([\cdots]\biggr\rvert_\mathrm{a}^\mathrm{b}\) represents the boundary terms resulting from the successive integration by parts. Here, \(u\) and \(v\) are arbitrary and sufficiently differentiable for functions so that the left and right sides are well defined. 

As an example, consider the linear differentiable operator
\begin{equation}
	L= a(x) \frac{d^2}{dx^2} + b(x)\frac{d}{dx} + c(x)
\end{equation}
To find \(L^*\), perform integration by parts on each term of the product \(vLu\) until there is no derivative of \(u\) within the integral. That is to say, integrate by parts twice on the first term, once on the second, and not at all on the third. Doing this, we are left with
\begin{equation}
	\begin{split}
		\intl vLu dx &= \intl (vau''+ vbu' + vc)dx\\
		&=(vau'+vbu)\biggr\rvert_\mathrm{a}^\mathrm{b} + \intl (-(va)'u'-(vb)'u+vcu)dx\\
		&=(vau'+vbu-(va)'u)\biggr\rvert_\mathrm{a}^\mathrm{b} + \intl ((va)''u-(vb)'u+vcu)dx\\
		&=(vau'+vbu-(va)'u)\biggr\rvert_\mathrm{a}^\mathrm{b} + \intl u((va)''-(bv)'+cv)dx.
	\end{split}
\end{equation}
From this, it is clear that 
\begin{equation}
	\begin{split}
		L^*v &= (Av)''-(Bv)'+Cv\\
		     &= (A'v+Av')'-B'v-Bv'+Cv\\
		     &= Av''+(2A'-B)v'+(A''-B'+C)
	\end{split}
\end{equation}
and so the formal adjoint of the second-order linear differential operator \(L\) must be of the form
\begin{equation}
	L^*=A\frac{d^2}{dx^2} + (2A'-B)\frac{d}{dx}+(A''-B'+C)
\end{equation}


If \(L^* = L\), then \(L\) is called formally self-adjoint. By comparing equations (2.2) and (2.5), we can see that for a second-order linear differentiable operator to be formally self-adjoint, \(A'\) must be equal to \(B\). You may notice that \(A''-B'+C\) must also be equal to C, but this is always true given that \(A'\) equals \(B\).

\begin{definition}
	If the boundary conditions on \(L\) are homogeneous\footnote{By homogeneous, we mean that each boundary condition is of the form \(B_j(u)=0\) and only contains terms that are 0 when \(u(x)=0\).}, then we can also define an adjoint operator, \(\mathcal{L}^*\), by the relation
	\begin{equation}
		(Lu,v) = (u,L^*v)
	\end{equation}
	where \((f,g)\) is the inner product of \(f\) and \(g\),
	\begin{equation}
		(f,g) = \int_a^bf(x)g(x)dx.
	\end{equation}
	This means adjoint operator \(\mathcal{L}^*\) consists of \(L^*\) and boundary conditions for which the boundary terms of the integral are zero. 
\end{definition}

\begin{example}
	Consider \(\mathcal{L}\) to consist of \(L=\frac{d}{dx}\) and the boundary condition \(u(0)=3u(1)\) over the interval \(0\leq x \leq 1\). Then
	\begin{equation}
		\begin{split}
			(Lu,v) &= \int_0^1u'vdx\\
			       &= (uv)\biggr\rvert_0^1 - \int_0^1 uv'dx\\
			       &= u(1)v(1)-u(0)v(0)+\int_0^1uL^*vdx\\
			       &= u(1)(v(1)-3v(0)) + \int_0^1uL^*vdx
		\end{split}
	\end{equation}
	Since the particular value of \(u(1)\) is not given, we must make \(v(1)-3v(0)\) equal zero, because choosing \(u(1)=0\) would undully restrict our solution. Therefore \(\mathcal{L^*}\) consists of \(L^*\) which is \(-\frac{d}{dx}\) and the boundary condition \(v(1) - 3v(0)=0\). As a final note, if \(\mathcal{L}=\mathcal{L}^*\), then \(\mathcal{L}\) is called self-adjoint.
\end{example}
