\section{The Adjoint Operator}
To determine the Green's function for a particular differential equation and it's boundary conditions we will need the formal adjoint operator. This operator, which we will call \(L^*\), can be found via repeated integration by parts. In general
\begin{equation}
	\int_a^b vLu dx = [\cdots]\biggr\rvert_a^b + \int_a^b uL^*v dx
\end{equation}
Here, \(u\) and \(v\) are completely arbitrary while being sufficiently differentiable for \(L\) and \(L*\) to exist. 

As an example, consider
\begin{equation}
	L= A(x) \frac{d^2}{dx^2} + b(x)\frac{d}{dx} + C(x)
\end{equation}
To find \(L^*\) perform integration by parts on the on each term of the product \(vLu\) a number of times equal to the order of the derivative that is a part of the term. To wit, twice on the first, once, on the second, and not at all on the third. Doing this, we are left with
\begin{equation}
	\begin{split}
		\int_a^b vLu dx &= \int_a^b (vAu''+ vBu' + vC)dx\\
		&=(vAu'+vBu)\biggr\rvert_a^b + \int_a^b (-(vA)'u'-(vB)'u+vCu)dx\\
		&=(vAu'+vBu-(vA)'u)\biggr\rvert_a^b + \int_a^b ((vA)''u-(vB)'u+vCu)dx\\
		&=(vAu'+vBu-(vA)'u)\biggr\rvert_a^b + \int_a^b u((vA)''-(Bv)'+Cv)dx
	\end{split}
\end{equation}
From this it is clear that 
\begin{equation}
	\begin{split}
		L^*v &= (Av)''-(Bv)'+Cv\\
		     &= (A'v+Av')'-B'v-Bv'+Cv\\
		     &= Av''+(2A'-B)v'+(A''-B'+C)
	\end{split}
\end{equation}
and so
\begin{equation}
	L^*=A\frac{d^2}{dx^2} + (2A'-B)\frac{d}{dx}+(A''-B'+C)
\end{equation}
