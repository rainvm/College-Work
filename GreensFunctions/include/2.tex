\section{The Adjoint Operator}
To determine the Green's function for a particular differential equation and it's boundary conditions we will need the formal adjoint operator. This operator, which we will call \(L^*\), can be found via repeated integration by parts. In general
\begin{equation}
	\int_a^b vLu dx = [\cdots]\biggr\rvert_a^b + \int_a^b uL^*v dx
\end{equation}
Here, \(u\) and \(v\) are completely arbitrary while being sufficiently differentiable for \(L\) and \(L*\) to exist. 

As an example, consider
\begin{equation}
	L= A(x) \frac{d^2}{dx^2} + b(x)\frac{d}{dx} + C(x)
\end{equation}
To find \(L^*\) perform integration by parts on the on each term of the product \(vLu\) a number of times equal to the order of the derivative that is a part of the term. That is to say, twice on the first term, once, on the second, and not at all on the third. Doing this, we are left with
\begin{equation}
	\begin{split}
		\int_a^b vLu dx &= \int_a^b (vAu''+ vBu' + vC)dx\\
		&=(vAu'+vBu)\biggr\rvert_a^b + \int_a^b (-(vA)'u'-(vB)'u+vCu)dx\\
		&=(vAu'+vBu-(vA)'u)\biggr\rvert_a^b + \int_a^b ((vA)''u-(vB)'u+vCu)dx\\
		&=(vAu'+vBu-(vA)'u)\biggr\rvert_a^b + \int_a^b u((vA)''-(Bv)'+Cv)dx
	\end{split}
\end{equation}
From this it is clear that 
\begin{equation}
	\begin{split}
		L^*v &= (Av)''-(Bv)'+Cv\\
		     &= (A'v+Av')'-B'v-Bv'+Cv\\
		     &= Av''+(2A'-B)v'+(A''-B'+C)
	\end{split}
\end{equation}
and so the formal adjoint of a second order linear differential operator \(L\) must be of the form
\begin{equation}
	L^*=A\frac{d^2}{dx^2} + (2A'-B)\frac{d}{dx}+(A''-B'+C)
\end{equation}

If \(L^*\) is found to be equal to \(L\) then \(L\) is called formally self adjoint. By comparing equations (2.2) and (2.5) we can see that for a second order linear differentiable operator to be formally self adjoint, \(A'\) must be equal to \(B\). You may notice that \(A''-B'+C\) must also be equal to C but this is always true given that \(A'\) equals \(B\).

If the boundary conditions on \(L\) are homogenous then we can also define an adjoint operator, \(\mathcal{L}^*\), by the relation
\begin{equation}
	(Lu,v) = (L^*v,u)
\end{equation}
where \((f,g)\) is the inner produce of \(f\) and \(g\)
\begin{equation}
	(f,g) = \int_a^bf(x)g(x)dx
\end{equation}
We now can understand that the adjoint operator \(\mathcal{L}^*\) must consist of \(L^*\) and boundary conditions to force the terms that come about from integrating by parts to be zero.

As an example of \(\mathcal{L}\) and \(\mathcal{L}^*\), consider \(\mathcal{L}\) to consist of \(L=\frac{d}{dx}\) and the boundary condition \(u(0)=3u(1)\) over the interval \(0\leq x \leq 1\). Then
\begin{equation}
	\begin{split}
		(Lu,v) &= \int_0^1u'vdx\\
		       &= (uv)\biggr\rvert_0^1 - \int_0^1 uv'dx\\
		       &= u(1)v(1)-u(0)v(0)+\int_0^1uL^*vdx\\
		       &= u(1)(v(1)-3v(0)) + \int_0^1uL^*vdx
	\end{split}
\end{equation}
Since the particular value of \(u(1)\) is not given, we must make \(v(1)-3v(0)\) equal zero, because choosing \(u(1)=0\) would undully restrict our solution. Therefore \(\mathcal{L^*}\) consists of \(L^*\) which is \(-\frac{d}{dx}\) and the boundary condition \(v(1) - 3v(0)=0\). As a final note, if \(\mathcal{L}=\mathcal{L}^*\), then \(\mathcal{L}\) is called self-adjoint.
