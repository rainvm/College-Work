\begin{example}
    We would like to show that the sequence
    \begin{equation*}
        w_k(x) = \begin{cases}
            k, &  0<x<\frac{1}{k}\\
            0, &x\leq 0 \text{ or } x \geq \frac{1}{k}
        \end{cases}
    \end{equation*}
    is a \(\d\)-sequence using Theorem \ref{th:assertion}. It is clear that \(w_k(x)\geq 0\)  for all \(x\) and \(w(x)=O(1/|x|^{1+\a})\) as \(|x| \rightarrow \inf\) with \(\a>0\) since it is zero when \(x \leq 0\) or \(x \geq 1/k\). Lastly, we must show that the area under \(w(x)\) is 1. Choosing \(w(x)\) so that \(kw(kx)=w_k(x)\),
    \begin{equation*}
        w(x)=\begin{cases}
            1,& 0<x<1\\
            0,& x\leq0 \text{ or } x \geq 1.
        \end{cases}
    \end{equation*}
    Clearly,
    \begin{equation}
        \begin{split}
            \intR w(x) dx &=1
        \end{split}
    \end{equation}
\end{example}

\begin{example}
    We would like to show that the sequence,
    \begin{equation}
        w_k(x) = \begin{cases}
            -k, &|x|<\frac{1}{2k}\\
            2k, &\frac{1}{2k} \leq |x| \leq \frac{1}{k}\\
            0, &|x|> \frac{1}{k},
        \end{cases}
    \end{equation}
    is a \(\d\)-sequence. Theorem \ref{th:assertion} does not apply in this case because \(w_k(x)\) is negative for some values of \(x\) so we should instead show 
    \begin{equation}
        \lim_{k\to \inf}\intR w_k(x) h(x) dx = h(0).
    \end{equation}
    To begin, we break up the integral 
    \begin{equation}
        \begin{split}
            \lim_{k\to \inf}\intR w_k(x) h(x) dx &= \lim_{k\to \inf}\int_{-\frac{1}{k}}^{-\frac{1}{2k}} 2kh(x)dx - \lim_{k\to \inf}\int_{-\frac{1}{2k}}^{\frac{1}{2k}} kh(x)dx \\
            &+ \lim_{k\to \inf}\int_{\frac{1}{2k}}^{\frac{1}{k}} 2kh(x)dx
        \end{split}
    \end{equation}
    
    Consider each integral. The mean value theorem (see Appendix A.2) shows that there exists a value, \(\x\), where \(-\frac1k < \x < -\frac{1}{2k}\), such that
    \begin{equation*}
        \begin{split}
            \lim_{k\to \inf}\int_{-\frac{1}{k}}^{-\frac{1}{2k}} 2kh(x)dx &= \lim_{k\to \inf}  \left( \left(-\frac{1}{2k} + \frac{1}{k}\right) 
            \cdot 2k\cdot h(\xi)\right) \\
            &=h(0).
        \end{split}
    \end{equation*}

    \begin{equation*}
        \begin{split}
            \lim_{k\to \inf}\int_{-\frac{1}{2k}}^{\frac{1}{2k}} -kh(x)dx &= \lim_{k\to \inf}  \left( \left(\frac{1}{2k} + \frac{1}{2k}\right) 
            \cdot -k\cdot h(\xi)\right) \\
            &=-h(0).
        \end{split}
    \end{equation*}

    \begin{equation*}
        \begin{split}
            \lim_{k\to \inf}\int_{\frac{1}{2k}}^{\frac{1}{k}} 2kh(x)dx &= \lim_{k\to \inf}  \left( \left(\frac{1}{k} - \frac{1}{2k}\right) \cdot 2k\cdot h(\xi)\right) \\
            &=h(0).
        \end{split}
    \end{equation*}
    The last step for each integral follows from the fact that as \(k \to \inf\), \(\xi\to 0\). Adding each integral shows that
    \begin{equation*}
        \lim_{k\to \inf}\intR w_k(x) h(x) dx = h(0).
    \end{equation*}
\end{example}