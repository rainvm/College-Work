\documentclass[12pt]{article}
\usepackage{array}
\usepackage{amsmath}
\usepackage{mathtools}
\usepackage{gensymb}
\usepackage{graphicx}
\usepackage{float}
\usepackage{caption}

\allowdisplaybreaks

\begin{document}
    \title{Physical Testing of Materials}
    \author{Ryan Coyne}
    \maketitle
    \section*{Introduction}
        This lab was done to learn about the physical testing of materials, and determine the identities of an unknown liquid and and unknown solid.
    \section*{Theory Discussion}
        The physical properties of a material are determined by it's chemical composition and therefore different materials will have different physical properties. A material can be identified by comparing measured physical properties to known physical properties of possible materials. 

        Solubility is the ability for one compound, the solute, to dissolve into another compound, the solvent. Dissolution occurs when the molecules of the solute are broken apart by the solvent into ions of it's components. Generally polar compounds are good at dissolving other polar compounds and non-polar compounds are good at dissolving other non-polar compounds.
        %TODO Expand

        Density is the measure of how much mass a material contains in a particular volume. This can be found by measuring the buoyant force on an object, or by measuring the mass and volume of the material and then dividing the measured mass by the measured volume.

        Melting point is the temperature at which the material transforms between a solid and a liquid. This temperature is dependent upon the pressure that the substance is under. As the pressure increases, the melting point tends to increase, and as the pressure decreases, the melting point tends to decrease.
    \section*{Procedure}
        \begin{enumerate}
            \item Fill a test tube half way with water.
            \item Fill the rest of the test tube with the unknown liquid.
            \item Cover the opening of the test tube with a finger.
            \item Vigorously shake the test tube for about thirty seconds.
            \item Check to to see if the liquid unknown has dissolved.
            \item Fill a test tube half way with ethanol.
            \item Repeat steps two through five with the test tube containing ethanol.
            \item Measure the mass of an empty beaker using a balance.
            \item Draw five milliliters of the unknown liquid into a five milliliter pipet.
            \item Add five milliliters of the liquid unknown to the empty beaker using a five milliliter pipet.
            \item Measure the mass of the beaker with five milliliters of the liquid unknown.
            \item Add another five milliliters of the liquid unknown to the beaker.
            \item Measure the mass of the beaker with ten milliliters of the liquid unknown.
            \item Add another five milliliters of the liquid unknown to the beaker.
            \item Measure the mass of the beaker with fifteen milliliters of the liquid unknown.
            \item Fill a test tube with water.
            \item Add the solid unknown to the test tube.
            \item Cover the opening of the test tube with a finger.
            \item Vigorously shake the test tube for about a minute.
            \item Record whether the solid unknown has dissolved or not.
            \item Fill a test tube with ethanol.
            \item Add the solid unknown to the test tube.
            \item Cover the opening of the test tube with a finger.
            \item Vigorously shake the test tube for about a minute.
            \item Record whether the solid unknown has dissolved or not.
            \item Place some of the solid unknown on a paper towel.
            \item Tap the opening of a capillary tube on the solid unknown several times.
            \item Tap the bottom of the capillary tube onto a table so the solid unknown falls to the bottom.
            \item If there is not a visible amount of the solid unknown at the bottom of the tube, repeat steps 26 and 27.
            \item Attach the capillary tube to the end of a thermometer using a rubber band.
            \item Fill a 500 ml beaker with water and place it on a hot plate.
            \item Place the thermometer in a clamp so that the end of the thermometer and the bottom of the capillary tube are in the water.
            \item Turn the thermometer on.
            \item Turn the hot plate to the highest setting.
            \item When the solid unknown begins to melt, record the temperature that is shown on the thermometer.
        \end{enumerate}

    \section*{Data and Calculations}
        \begin{table}[H]
            \centering
            \caption{Solubility}
            \begin{tabular}{c|cc}
                &Water&Ethanol\\
                \hline
                Liquid Unknown & No & Yes\\
                Solid Unknown & No & Yes
            \end{tabular}
        \end{table}
        \begin{table}[H]
            \centering
            \caption{Mass of Liquid Unknown}
            \begin{tabular}{c|ccc}
                & Trial 1 & Trial 2 & Trial 3\\
                \hline
                Empty Beaker (g) & 37.84 & 41.87 & 45.61\\
                Beaker and 5.0ml of liquid unknown (g) & 41.87 & 45.61 & 49.47 \\
                \hline
                5.0 ml of liquid unknown (g) & 4.03 & 3.74 & 3.86
            \end{tabular}
        \end{table}
        \begin{center}
            Melting Point of Solid Unknown: 47.7\(\degree\)C\\ 
        \end{center}
        \begin{align*}
            \rho_1 & = \frac{4.03~\mathrm{g}}{5.0~\mathrm{ml}}\\
            & = 0.806~\mathrm{g/ml}\\
            \rho_2 & = \frac{3.74~\mathrm{g}}{5.0~\mathrm{ml}}\\
            &= 0.748 ~\mathrm{g/ml}\\
            \rho_3 & = \frac{3.86~\mathrm{g}}{5.0~\mathrm{ml}}\\
            &= 0.772 ~\mathrm{g/ml}\\
            \overline{\rho} &= \frac{0.806~\mathrm{g/ml}+0.748~\mathrm{g/ml}+0.772~\mathrm{g/ml}}{3}\\
            &= 0.775~\mathrm{g/ml}
        \end{align*}
    \section*{Conclusion}
        The goal this lab was to see how physical testing can be used to identify materials. We found that the unknown liquid was cyclohexane and the unknown solid was benzophenone. 

        When drawing the liquid unknown into the pipet for the third mass measurement, the liquid was drawn into the bulb. Although we attempted to empty the bulb as thoroughly as possible before , it may have caused more than five milliliters to be added to the beaker which would have added error to the measurement. 
\end{document}