\documentclass[12pt]{article}
\usepackage{array}
\usepackage{amsmath}
\usepackage{mathtools}
\usepackage{gensymb}
\usepackage{graphicx}
\usepackage{float}
\usepackage{caption}

\allowdisplaybreaks

\begin{document}
    \title{Lab 1 Questions}
    \author{Ryan Coyne}
    \maketitle
    \begin{itemize}
        \item[1] Precision refers to how close multiple measurements are to each other. Accuracy refers to how close measurements are to the actual value.
        \item[2] A counting number is an integer the describes the exact number of items which there cannot be fractions of. A measuring number can be any real number which has a unit associated with it and is not exact.
        \item[3] Using 50 pennies would have changed the number of significant figures because the volume measurement of the pennies would have increased above 10 and therefore would have had two significant figures instead of one.
        \item[4] Archimedes principle says that the weight of the fluid displaced by an object in a fluid is equal to the buoyant force on the object. It also tells us that the volume of the fluid displaced is equal to the volume of the object submersed. This can be used in the construction of ships to make sure that it floats and that only the correct parts of the ship are under water.
        \item[5] \(\frac{x~\mathrm{lb}}{\mathrm{qt}} = \frac{1.0~\mathrm{g}}{\mathrm{ml}} \cdot \frac{1~\mathrm{lb}}{454~\mathrm{g}} \cdot \frac{1000~\mathrm{mL}}{1~\mathrm{L}} \cdot \frac{1.06~\mathrm{qt}}{1~\mathrm{L}} = 2.3~\mathrm{lb/qt}\)
    \end{itemize}
\end{document}