\documentclass[12pt]{article}
\usepackage{array}
\usepackage{amsmath}
\usepackage{mathtools}
\usepackage{gensymb}
\usepackage{graphicx}
\usepackage{float}
\usepackage{caption}

\allowdisplaybreaks

\begin{document}
    \title{Oxidation Reduction \& \%Yield}
    \author{Ryan Coyne}
    \maketitle
    \section*{Introduction}
        The purpose of this lab is to learn about \%yield and to perform an oxidation reduction reaction.
    \section*{Theory Discussion}
        An oxidation reduction reaction, also known as a redox reaction, is a type of chemical reaction in which the oxidation states of atoms in the reaction change. The oxidation number represents the difference in protons and electrons in the atom. Oxidation is the loss of electrons which results in an increase in the oxidation number and reduction is the gain of electrons which results in a decrease in the oxidation number. 

        \%Yield is the percentage of the expected result that was obtained after the reaction. This is found by dividing the actual amount of final product by the expected amount of final product. \%Yield is typically less than 100\% because of inaccuracy in measurements and because it is nearly impossible to capture every part of the resultant substance.
    \section*{Procedure}
        \begin{enumerate}
            \item Fill a 600 ml beaker to the 100ml mark and heat on a hot plate to a boil.
            \item Weigh a paper towel.
            \item Add between 7.50 g and 8.50 g of copper (II) sulfate.
            \item Slowly add the copper (II) sulfate to the beaker and stir until it is all dissolved.
            \item Weight two nails.
            \item Carefully add the two nails to the solution and let it react for 15 minutes using slow stirring and low heat.
            \item Remove both nails with tongs.
            \item Carefully scrape any copper off the nails into the beaker.
            \item Squirt the nails with water over the sink and then wipe with a paper towel.
            \item Weigh the nails.
            \item Weigh a filter paper and watch glass.
            \item Filter the solution. Make sure to get all of the copper from the beaker onto the filter paper.
            \item After the filter paper has completely dried, weigh the watch glass, filter paper, and copper.
        \end{enumerate}
    \section*{Calculations}
        \begin{table}[h]
            \centering
            \begin{tabular}{lc}
                Mass of the iron nails before reaction: & 3.11 g\\
                Mass of iron nails after reaction: & 2.62 g\\
                Mass of iron reacted: & 0.49 g\\
                Mass of filter paper \& watch glass: & 43.51 g\\
                Mass of copper, filter paper \& watch glass: & 44.31 g\\
                Mass of copper obtained: & 0.80 g
            \end{tabular}
        \end{table}
        \begin{equation*}
          2 \mathrm{Fe} + 3 \mathrm{CuSO}_4 \rightarrow \mathrm{Fe}_2(\mathrm{SO}_4)_3 + 3 \mathrm{Cu}
        \end{equation*}
        \(
            \text{Expected Yield: }0.49~\mathrm{g~Fe}\cdot \frac{1~\mathrm{mol}}{55.845~\mathrm{g}}\cdot \frac{3~\mathrm{mol~Cu}}{2~\mathrm{mol~Fe}} = 0.0132~\mathrm{mol~Cu}\\
            \text{Actual Yield: } \frac{0.80\text{ g Cu}}{63.564\text{ g/mol}} = 0.0126\text{ mol Cu}\\
            \text{\% Yield}: \frac{0.0126}{0.0132} = 96\%
        \)
    \section*{Conclusion}
        The goal of this lab was to perform a redox reaction with iron nails and copper (II) sulfate and then determine the \%yield of the reaction. The expected yield was 0.0.132 moles of copper and the actual yield was 0.126 moles of copper. This is a 96\% yield. We saw how reactions yield less than the theoretically expected amount of product.

        In this lab error may have resulted because there was residual copper left on the nails, not all of the copper may have been completely filtered out of the water, some copper may have been left in the filter when the paper was removed, and when removing the filter paper from the filter, copper fell off of the paper and onto the table and floor. We retrieved most of this copper but some may have been left behind.
\end{document}