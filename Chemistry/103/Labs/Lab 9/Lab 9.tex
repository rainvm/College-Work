\documentclass[12pt]{article}
\usepackage{array}
\usepackage{amsmath}
\usepackage{mathtools}
\usepackage{gensymb}
\usepackage{graphicx}
\usepackage{float}
\usepackage{caption}

\allowdisplaybreaks

\begin{document}
    \title{Electron Flame Test Lab}
    \author{Ryan Coyne}
    \maketitle
    \begin{enumerate}
        \item When energy in the form of heat is added to the substances, it excites their electrons to a higher energy level. When the electrons return to the ground state, they release photons of a particular wavelength.
        \item A quanta is a discrete amount of a physical property, such as energy, that an entity can have.
        \item Wavelength is the distance between a point in one cycle of a wave and an identical point in the next cycle. Frequency is the number of cycles that pass a point per unit of time.
        \item The color of light is determined by it's wavelength. The wavelength determines how strongly light interacts with each of three types of cones in our eyes and our brain interprets that as color.
        \item
    \end{enumerate} 
    \begin{table}[h]
        \centering
            \begin{tabular}{c|ccc}
                & Cu(II)CO\(_3\) & LiCl & CoCl\\
                \hline
                Color & Green & Red & Orange \\
                Frequency & 5.6\(\times 10^{14}\) Hz & 4.3\(\times 10^{14}\) Hz & 4.9\(\times 10^{14}\) Hz
            \end{tabular}
    \end{table}
    \begin{enumerate}
        \setcounter{enumi}{5}
        \item 
    \end{enumerate}
    \begin{table}[h]
        \centering
            \begin{tabular}{c|ccc}
                & Cu(II)CO\(_3\) & LiCl & CoCl\\
                \hline
                Energy & \(3.7 \times 10^{-19}\) J & \(2.8 \times 10^{-19}\) J & \(3.2 \times 10^{-19}\) J
            \end{tabular}
    \end{table}
    \begin{enumerate}
        \setcounter{enumi}{6}
        \item Electrons exhibit properties of particles such as having mass and the ability to collide. They also exhibit properties of waves such as interference and diffraction.
        \item Niels Bohr was a Danish physicist born in 1885. He developed a model of the atom that has the electrons revolving in stable orbits around the nucleus at discrete energy levels. Now we know that the electrons do not orbit in stable circular obits but they do exist at discrete energy levels.
        \item The bunsen burner was invented by Robert Bunsen and Peter Desaga. Beaker was Dr. Bunsen's assistant in the Muppets show.
        \item My favorite was the CuCO\(_3\).
    \end{enumerate}

\end{document}