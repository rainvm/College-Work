\documentclass[12pt]{article}
\usepackage{array}
\usepackage{amsmath}
\usepackage{mathtools}
\usepackage{gensymb}
\usepackage{graphicx}
\usepackage{float}
\usepackage{caption}
\usepackage{multirow}

\allowdisplaybreaks

\begin{document}
    \title{Design of Experiment}
    \author{Ryan Coyne}
    \maketitle
    \begin{table}[h]
        \centering
        \begin{tabular}{cc|cccc}
            &&\multicolumn{2}{c}{Cold} & \multicolumn{2}{c}{Hot}\\
            &&5 min & 10 min & 5 min & 10 min\\
            \hline
            \multirow{2}{*}{No Bleach} 
            & No Soap & 5 & 3 & 4 & 3\\
            & Soap & 5 & 2 & 3 & 2\\
            \multirow{2}{*}{Bleach} 
            & No Soap & 3 & 2 & 2 & 1\\
            & Soap & 3 & 1 & 2 & 1
        \end{tabular}
        \caption*{1=Best, 5=Worst}
    \end{table}
    
    From this lab I learned that tables are useful for designing experiments that test every combination of variables. I also learned that grape juice stains on a shirt continue to set over years because the shirt had been stained a couple of years ago before the pandemic the stains are usually fully removed from this shirt in this lab but this time it was difficult to see much of a difference.
    
    \begin{table}[H]
        \centering
        \caption*{Example DOE table for cooking a pizza}
        \begin{tabular}{cc|cccc}
            &&\multicolumn{2}{c}{Pepperoni} &\multicolumn{2}{c}{BBQ Chicken}\\
            &&Sicilian & New York & Sicilian & New York\\
            \hline
            \multirow{2}{*}{Oven}
            & Mozzarella & 1 & 2 & 3 & 4\\
            & Provolone & 5 & 6 & 7 & 8\\
            \multirow{2}{*}{Stovetop} & Mozzarella & 9 & 10 & 11 & 12\\
            & Provolone & 13 & 14 & 15 & 16
        \end{tabular}
    \end{table}
\end{document}