\documentclass[12pt]{article}
\usepackage{array}
\usepackage{amsmath}
\usepackage{mathtools}
\usepackage{gensymb}
\usepackage{graphicx}
\usepackage{float}
\usepackage{caption}

\allowdisplaybreaks

\begin{document}
    \title{Chemical Formulas}
    \date{June 12, 2022}
    \author{Ryan Coyne}
    \maketitle
    \section*{Introduction}
        In this lab we reacted zinc with hydrochloric acid to learn about how chemical reactions can be used to determine the chemical composition of compounds.
    \section*{Theory Discussion}
        In chemistry when analyzing reactions we are working with ratios of the number atoms and molecules not masses so it is important to be able to convert from mass to moles and vice versa which we can do using moles. A mole is a SI unit. The mole has gone through several changes to the way it is defined over the years. It has been defined in terms of hydrogen, oxygen, and most recently carbon-12, but it is currently defined as exactly 6.02214076 \(\times\) 10\(^{23}\) of something, usually atoms or molecules. Moles are used in chemistry because it's not possible to directly count the number of atoms or molecules in a sample, so instead the mass is measured and the number of moles of atoms or molecules can be found from that measurement if it is known which atoms or molecules comprise the sample. This can be done because each element has a known molar mass, that is the mass of one mole of that element. It is possible to find the empirical formula of a compound by analyzing the ratio of moles of the elements in the compound.
    \section*{Procedure}
    \begin{enumerate}
        \item Measure the mass of a clean, dry evaporating dish using a balance.
        \item Add 0.40 g to 0.60 g of zinc to the evaporating dish.
        \item Record the mass of the zinc that was added.
        \item Under a fume hood add 10 ml of HCl solution (6 M) to the evaporating dish.
        \item When the mixture has finished bubbling remove it from the fume hood.
        \item Heat for 5 minutes over a water bath.
        \item Remove the water bath after the 5 minutes is up.
        \item Heat the dish directly on the hot plate until it turns to a solid.
        \item Do not overheat because the zinc chloride may sublime.
        \item Cool the evaporating dish on the bench for 2-3 minutes.
        \item Measure the mass of the zinc chloride using a balance before it can absorb water from the air.
    \end{enumerate}
    \section*{Calculations}
        \begin{tabular}{ll}
            Mass of Empty Evaporating Dish:& 113.97 g\\
            Mass of Zinc + Evaporating Dish:& 114.48 g\\
            Mass of Zinc:& 113.97 g - 114.48 g = 0.51 g\\
            Mass of Zinc Chloride + Evaporating Dish:& 114.92 g\\
            Mass of Zinc Chloride:& 114.92 g - 113.97 g = 0.95 g\\
            Mass of Chlorine:& 0.95 g - 0.51 g = 0.44 g\\
            Moles of Zinc:& 0.51 g / 65.38 g/mol = 0.0078 mol\\
            Moles of Chlorine:& 0.44 g / 35.45 g/mol = 0.012 mol\\
            Ratio of Moles of Chlorine to Moles of Zinc:& 0.0078 mol / 0.012 mol = 1.6\\
            Rounded Ratio:& 2\\
            Chemical Formula:& ZnCl\(_2\)\\
        \end{tabular}
    \section*{Conclusion}
        The goal of this lab was to use a balance and a chemical reaction to determine the chemical formula of zinc chloride. The chemical formula was determined to be ZnCl\(_2\) which accurate to the known formula for zinc chloride. Integral to this process was the concept of moles. Without a concept of moles and known molar masses, it would not have been possible to determine the chemical formula this way which illustrates just how important moles are to chemistry. 

        Error in this experiment likely arose from impurities in the reagents and in the result. The zinc chloride may have not been perfectly dry when weighed at the end, because some of the water and HCl may not have been completely boiled away and the zinc chloride may have begun absorbing water from the surrounding air before it's mass was measured.
\end{document}