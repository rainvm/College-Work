\documentclass[12pt]{article}
\usepackage{array}
\usepackage{amsmath}
\usepackage{mathtools}
\usepackage{gensymb}
\usepackage{graphicx}
\usepackage{float}
\usepackage{caption}

\allowdisplaybreaks

\begin{document}
    \title{Determining the Rate Constant of a Reaction}
    \author{Ryan Coyne}
    \maketitle
    \section*{Introduction}
        The purpose of this lab is to learn how the concentration reagents affects the speed of a reaction by calculating rate constants for the reaction of Zn with three different concentrations of HCl. 
    \section*{Theory Discussion}
        The rate of a reaction is the moles reacted per unit of time. This rate is affected by a number of factors including temperature, concentration, and surface area. The rate of a reaction can be calculated using the equation:
        \begin{equation*}
            \text{Rate} = k[A]^m[B]^n
        \end{equation*}
        where \(k\) is the rate constant and is known for the particular conditions under which the reaction is taking place, \([A]\) is the concentration of the first reactant, \([B]\) is the concentration of the second reactant, and \(m\) and \(n\) are known for the reaction. 
    \section*{Procedure}
    \begin{enumerate}
        \item Weigh an empty 125 ml Erlenmeyer flask.
        \item Add between 0.40 and 0.60 g of zinc to the flask.
        \item Put on safety goggles.
        \item Add 25 ml of 6 M HCl solution to the flask.
        \item Measure the time it takes for the gas to stop evolving.
        \item Repeat steps 1-5 using a 4.8 M HCl solution.
        \item Repeat steps 1-5 using a 3.6 M HCl solution.
    \end{enumerate}

    \section*{Calculations}
    \begin{table}[h!]
        \centering
        \begin{tabular}{c|ccc}
            & 6 M HCl & 4.8 M HCl & 3.6 M HCl \\
            \hline
            Weight of empty flask &  0.64 g &  0.64 g &  0.64 g \\
            Weight of flask with zinc &  1.19 g &  1.16 g &  1.20 g \\
            Mass of zinc &  0.55 g &  0.52 g &  0.56 g \\
            Moles of zinc &  0.0084 & 0.0080 & 0.0086 \\
            Reaction time & 6:04 & 6:32 & 7:38 \\
            Reaction rate & \(2.3 \times 10^{-5}\) moles/s & \(2.0 \times 10^{-5}\) moles/s & \(1.9 \times 10^{-5}\) moles/s \\
            Moles of HCl & 0.017 & 0.016 & 0.017 \\
            K value & 506 \(\mathrm{moles^{-3}s^{-1}}\) & 633 \(\mathrm{moles^{-3}s^{-1}}\) & 434 \(\mathrm{moles^{-3}s^{-1}}\)\\
        \end{tabular}
    \end{table}

    \section*{Conclusion}
        The goal of this lab was to determine the rate constant for the reaction of Zinc and Hydrochloric acid. We performed the experiment with three different concentrations of HCl and calculated a rate constant for each. These were 506, 633, and 434. 

        There was a lot of error involved in this experiment which can be seen in the disparity between the rate constants. We determined the end of the reaction by visual inspection of the bubbles of \(\mathrm{H_2}\) generated and there was variation in the sizes and shapes of the zinc pieces which causes variation in the surface area of the reaction. 
\end{document}