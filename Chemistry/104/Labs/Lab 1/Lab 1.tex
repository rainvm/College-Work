\documentclass[12pt]{article}
\usepackage{array}
\usepackage{amsmath}
\usepackage{mathtools}
\usepackage{gensymb}
\usepackage{graphicx}
\usepackage{float}
\usepackage{caption}

\allowdisplaybreaks

\begin{document}
    \title{Materials Science}
    \author{Ryan Coyne}
    \maketitle
    \begin{enumerate}
        \item Polymers are long chains of repeating molecular units that are bonded together into one molecule. They have covalent intramolecular bonding and Van Der Waal intermolecular bonding.
        \item Thermoplastics have low melting points and can be melted and molded without compromising the material. Thermosets have high melting points and will combust before melting, so these can't be reformed after setting. Thermosets tend to be harder and more rigid than thermoplastics. Thermoplastics are made of long polymer chains. Thermosets have long chains that are cross-linked by chemical bonds. Plastic bottles are made from thermoplastics and car tires are made from thermosets. thermoplastics are easier to make parts from because they are stronger and  can be recycled.
        \item Thermoforming would not work for thermosets because they have a very high melting point and will combust before they melt.
        \item Thermosets take longer to cure.
        \item Charles Goodyear made the first thermosetting polymer, vulcanized rubber, in 1839.
        \item Thermoplastic beads are added into a tube with a heated and rotating screw inside. The beads are melted and simultaneously moved to the end of the screw. When a sufficient ammount of melted thermoplastic is accumulated at the end of the screw, it is pushed forward and the plastic is injected into the mold.
        \item Ceramics are made of polar molecules that are tightly bonded together with electrostatic intermolecular bonds to form crystalline structures.
        \item Clay is primarily composed of $\mathrm{SiO_2}$ and $\mathrm{Al_2O_3}$.
        \item Sintering uses high heat to fuse the molecules of the clay together. The high heat gives the molecules the energy needed to change their structure and results in a tightly packed crystalline structure called a ceramic.
        \item Beaker is very cool!
    \end{enumerate}
\end{document}