\documentclass[12pt]{article}
\usepackage{array}
\usepackage{amsmath}
\usepackage{mathtools}
\usepackage{gensymb}
\usepackage{graphicx}
\usepackage{float}
\usepackage{caption}

\allowdisplaybreaks

\begin{document}
    \title{Twisting the Dragon's Tail Questions}
    \author{Ryan Coyne}
    \maketitle
    \begin{enumerate}
        \item Background radiation is the level of ionizing radiation that is always present in the environment that is not due to artificially added sources of radiation. Background radiation is measured in microsieverts per hour.
        \item  He had been putting uranium on top of black paper which was on top of film in the sun and saw that the film was exposed where the uranium had been. He took this to mean that the uranium was absorbing and re-emitting energy from the sun. When it rained he put the experiment he was planning to do that day in a drawer and noticed that it was still exposed around the uranium which meant that it had nothing to do with the sun and the energy was coming from the uranium itself.
        \item Marie Curie discovered radium and it's radioactivity which seemed to contradict the theory of conservation of energy. She also discovered that thorium and polonium, an element that she and her husband discovered and named, are also radioactive. She also discovered that tumor-forming cells were destroyed faster by radiation than healthy cells, laying the groundwork for cancer radiation therapy.
        \item \(\mathrm{E=mc^2}\) is Einstein's famous equation. E is energy, m is mass, and c is the speed of light.
        \item X-Rays are produced with a vacuum tube that accelerates electrons to a high velocity which collide with an annode and create the X-Rays.
        \item Alpha particles are emitted when uranium decays into thorium.
        \item Enriched uranium is a sample of uranium that has an artificially increased ammount of uranium-235.
        \item Nuclear fision involves splitting an atom and nuclear fusion involves combining atoms. Fision is used in atomic bombs.
        \item A chain reaction occurs when one atom is split by a neutron and that atom releases two or more neutrons which go on to split other atoms which also release neutrons which split further atoms and so-on.
        \item If it was the baths that Derek Muller went to in the documentary then it would probably be safe, but it would not be a good idea at higher concentrations. It probably wouldn't make his hair color change but it might make it fall out.
    \end{enumerate}
\end{document}