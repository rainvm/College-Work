\documentclass[12pt]{article}
\usepackage{array}
\usepackage{amsmath}
\usepackage{mathtools}
\usepackage{gensymb}
\usepackage{graphicx}
\usepackage{float}
\usepackage{caption}
\usepackage{stackrel,amssymb}

\allowdisplaybreaks

\begin{document}
    \title{Synthesis of Asprin}
    \author{Ryan Coyne}
    \maketitle
    \section*{Introduction}
        The purpose of this lab lab was to produce asprin and determine the \% yield of our product.
    \section*{Theory Discussion}
        The expected amount of product can be calculated from the initial amount of the reagents using the chemical equation which in this case is:
        \begin{equation*}
            \mathrm{C_7H_6O_3} + \mathrm{C_4H_6O_3} \leftrightarrows \mathrm{C_9H_8O_4} + \mathrm{C_2 H_4O_2}
        \end{equation*}
        The percent yield is the percentage of the expected prodcut that was actually obtained and can be calculated using the formula:
        \begin{equation*}
            \%~\text{yield} = \frac{\text{Theoretical Yield}}{\text{Actual Yield}} \cdot 100\%
        \end{equation*}
    \section*{Procedure}
        \begin{enumerate}
            \item Fill a 600 ml beaker half full of water.
            \item Add boiling chips to the beaker.
            \item Weigh an empty 125 ml Erlenmeyer flask.
            \item Add 2.85 - 3.15 g of salicylic acid to the flask.
            \item Under a fume hood, with safe glasses on, carefully measure 6.0 ml of acetic anhydride in a 10 ml graduated cylinder and add it to the flask.
            \item Swirl the mixture for 20-30 seconds and add 5 drops of \(\mathrm{H_2SO_4}\).
            \item Heat the flask for 10 minutes in a water bath at a temperature between 80 and 90 degrees C.
            \item Remove the flask and cool to room temperature.
            \item Add 40 ml of water to the flask and put it in an ice bath.
            \item Weigh a filter paper and watch glass.
            \item When the asprin has fully recrystallized, filter it.
            \item Put the filter paper and asprin on the watch glass.
            \item Wait for the asprin to dry.
            \item Weigh the asprin, fitler paper, and watch glass.
            \item Put a little ferric chloride solution into a test tube and add a bit of asprin. If the color changes to purple, this indicates the presence of salicylic acid.
        \end{enumerate}
    \section*{Calculations}
        \begin{table}[h!]
            \centering
            \begin{tabular}{c|c}
                & Mass (g) \\
                \hline
                Salicylic Acid & 2.85 \\
                Filter Paper & 0.20 \\
                Watch Glass & 42.43 \\
                Asprin, Filter Paper, \& Watch Glass & 44.53 
            \end{tabular}
        \end{table}
        \begin{align*}
            \\\\\\
            \text{Theoretical Yield} &= 2.85 \mathrm{g} \cdot \frac{1 \mathrm{mol}}{132.13 \mathrm{g}} \cdot \frac{180.17 \mathrm{g}}{1 \mathrm{mol}}\\
            &= 3.89 \mathrm{~ g ~ of ~ C_9H_8O_4}\\
            \text{Actual Yield} &= 44.53 ~ \mathrm{g} - 0.2 ~ \mathrm{g} - 42.43 ~ \mathrm{g}\\
            &= 1.90~\mathrm{g}\\
            \mathrm{\%~yield} &= \frac{1.90}{3.89}\\
            &= 48.8 \%
        \end{align*}
    \section*{Conclusion}
        The goal of this lab was to synthesize asprin and calculate the \% yield. Our \% yield was 48.8\%, but the ferric chloride did not turn purple which indicates that our product was pure. We assumed that the salicylic acid was the limiting reagent which may not be the case since we did not calculate the ammount of acetic anhydride that was used. The asprin may have also not completely recrystallized out of the solution. When transfering the asprin and filter paper from the filter to the watchglass, dropped all of it and were probably not able to completely collect the asprin. back onto the watchglass.
\end{document}