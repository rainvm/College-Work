\documentclass[12pt]{article}
\usepackage{array}
\usepackage{amsmath}
\usepackage{mathtools}
\usepackage{gensymb}
\usepackage{graphicx}
\usepackage{float}
\usepackage{caption}

\allowdisplaybreaks

\begin{document}
    \title{Making Biodiesel Fuel Questions}
    \author{Ryan Coyne}
    \maketitle
    \begin{enumerate}
        \item Biodiesel fuel is made from vegetable oil and regular diesel is made from petroleum; as a result, biodiesel is much less toxic than regular diesel fuel. Biodiesel also usually results in slightly less engine power than petroleum diesel.
        \item No, the fuel pump would struggle to move the denser diesel, and it would clog up the fuel filter. It will also clog the fuel injectors causing the engine to seize up.
        \item Combustion occurs when fuel is used in a vehicle. Oxygen is needed in addition to the fuel. Water and carbon dioxide are formed.
        \item The reaction in \#3 is exothermic.
        \item A catalyst is not consumed in the reaction but a reactant is.
        \item Titration is the processes of measuring the quantity or concentration of a substance by adding small measured quantities of a substance that will react with it and determining when all of the original substance has reacted.
        \item Oil is non-polar. Polarity is determined by the relative electronegativity of bonded atoms.
        \item Soap has a polar end composed of oxygen atoms and a metal cation with a non-polar tail.
        \item Viscosity measures the internal friction of a fluid. The viscosity of an organic compound increases with it's molecular mass.
        \item I don't think he uses biodiesel because it's probably simpler to get petroleum diesel from a gas station.
    \end{enumerate}

\end{document}