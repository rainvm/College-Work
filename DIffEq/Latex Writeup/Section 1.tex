\documentclass{report}
\usepackage{graphicx}
\usepackage{array}
\usepackage{amsmath}
\usepackage{mathtools}
\usepackage{gensymb}
\usepackage{float}
 

\begin{document}
    \chapter{Introduction To Differential Equations}
    \section{Definitions and Terminology}
        \paragraph{}Ordinary differential equations, ODEs, have only one independent variable but may have one or more 
        dependant variables. The equations 
        \begin{center}
            \(\frac{dy}{dx}+5y=e^x\), \, \(\frac{d^2y}{dx^2}-\frac{dy}{dx}+6y = 0\), and \(\frac{dx}{dt}+
            \frac{dy}{dt} = 2x+y\)
        \end{center}
        are all examples of ordinary differential equations.\\
        
        Partial differential equations have more than one independent variable and therefore involve partial 
        derivatives. The equations 
        \begin{center}
            \(\frac{\partial^2u}{\partial x^2}+\frac{\partial^2u}{\partial y^2} = 0\), and 
            \(\frac{\partial^2u}{\partial y^2} = \frac{\partial^2u}{\partial t^2} - 2\frac{\partial u}{
                \partial t}\)
        \end{center}
        are examples of partial differential equations. This text will only discuss ordinary DEs.\\
        
        The order of a differential equation is the order of the highest order derivative in the DE. For example, the 
        equation
        \begin{center}
            \(5y''+y'-10xy=0\)
        \end{center}
        is a second order differential equation.\\

        Linear ODEs are equations if the dependant variable(s) are linear. In other words the 
        coefficients must only contain the dependant variable and/or constants and no functions such as 
        sine or logarithms are applied to the dependant variable(s). Linear ODEs have the form 
        \begin{center}
            \(a_n(x)y^{(n)}+a_{n-1}(x)y^{(n-1)} + ... + a_1(x)y'+a_0(x)y-g(x) = 0\)
        \end{center}
        The equations 
        \begin{center}
            \(y''-3y=0\), \, \(x^3\frac{d^3y}{dx^3}+x\frac{dy}{dx}-5y=e^x\), and \((y-x)dx+4xdy=0\)
        \end{center}
        are examples of linear ODEs and the equations
        \begin{center}
            \((1-y)y'+2y=e^x\), \(\frac{d^2y}{dx^2}+\sin y = 0\), and \(\frac{d^4y}{dx^4}+y^2 = 0\)
        \end{center}
        are nonlinear ODEs.\\

        The solution of a differential equation is a function \(\phi\) such that it reduces the ODE to an identity when 
        substituted into the equation. In other words \(\phi\) has at least \(n\) derivatives for which
        \begin{center}
            \(F(x, \phi(x), \phi'(x),...,\phi^{(n)}(x)) = 0\).
        \end{center}
        The solution will often be denoted by \(y\) or \(y(x)\).\\

        A relation \(G(x,y)=0\) is said to be an implicit solution of an ODE provided there is at least one function 
        \(\phi\) that satisfies the relation as well as the differential equation.\\

        If a differential equation has n solutions, \(y_1 , \, y_2, ..., \, y_n\), then we can say a general solution 
        is \(y = c_1y_1 + c_2y_2 + ... + c_ny_n\) where \(c_1\) to \(c_n\) are constant coefficients. A solution to 
        a differential equation that is free of unknown coefficients is called a particular solution. 
    \section{Initial-Value Problems}
        \paragraph{}Sometimes we want to find a solution \(y(x)\) that satisfies certain conditions, these are called 
        initial-value problems. The conditions for an initial value problem are called initial conditions. 
        For an initial value problem involving an ODE of order \(n\) there are \(n\) initial conditions
        \begin{center}
            \(y(x_0)=y_0,\, y'(x_0) = y_0, . . . ,\, y^{(n-1)}(x_0)=y_{n-1}\).
        \end{center}
        To find the particular solution(s) to an IVP we plug the given conditions into the general solution and 
        it's applicable derivatives and then solve the system of equations to find the constant coefficients.
\end{document}