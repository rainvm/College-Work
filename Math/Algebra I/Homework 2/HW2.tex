\documentclass[12pt]{article}

\newcounter{results}
\newcounter{questions}

\def\neg{{\sim}}
\def\Z{\mathbb{Z}}
\def\N{\mathbb{N}}
\def\R{\mathbb{R}}
\def\Q{\mathbb{Q}}
\def\E{\mathbb{E}}
\def\qed{\(\blacksquare\)}
\newcommand{\result}[1]{\stepcounter{results}{\bfseries Result \arabic{results}}: #1}
\newcommand{\question}[1]{\stepcounter{questions}{\bf \arabic{questions}}: #1}
\newenvironment{proof}[2][Proof]{\result{#2}\begin{trivlist} 
    \item[\hskip \labelsep {\sc #1:}]}{\qed\end{trivlist}}
\usepackage{array}
\usepackage{amsmath}
\usepackage{amssymb}
\usepackage{mathtools}
\usepackage{textcomp}
\usepackage{gensymb}
\usepackage{graphicx}
\usepackage{float}
\usepackage{caption}
\usepackage{amsfonts}
\usepackage[margin=1in]{geometry}

\title{Homework 2}
\author{Ryan Coyne}

\def\ZN{\mathbb{Z}_N}


\begin{document}
    \maketitle

    \begin{proof}{Let (\(G, \circ\)) be a group, and let \(g, h\in G\). Then 
        \begin{equation*}
            (g\circ h)^{-1} = h^{-1} \circ g^{-1}.
        \end{equation*}}
        First, we compose \((g \circ h)^{-1}\) with \((g\circ h)\). This gives us, \[(g \circ h)^{-1}\circ(g\circ h) = e.\] Next compose \(g^{-1}\circ h^{-1}\) with \((g\circ h)\). This gives us \begin{align*}
            (h^{-1}\circ g^{-1}) \circ (g\circ h) &= h^{-1} \circ (g^{-1} \circ g) \circ h\\
            & = h^{-1} \circ h\\
            & = e.
        \end{align*}
        Now, since \((g \circ h)^{-1}\circ(g\circ h) = (h^{-1}\circ g^{-1}) \circ (g\circ h)\), it follows that \((g \circ h)^{-1} = h^{-1}\circ g^{-1}\).
    \end{proof}

    \begin{proof}{Let \(\mathbb{Z}_N = {0, 1, 2, \cdots, N-1}\). Show that the operation \(\circ: \mathbb{Z}_N \times \ZN \rightarrow \ZN\) defined by \begin{equation*}
        a \circ b := \begin{cases}
            a + b,  &a + b < N\\
            a + b - N, & a + b \geq N
        \end{cases}
    \end{equation*}
    is associative. i.e. Show that for any \(a, b, c \in \ZN\), we have 
    \begin{equation*}
        (a\circ b) \circ c = a \circ (b \circ c)
    \end{equation*}}
    
    \underline{Case 1:} 
    \end{proof}
\end{document}