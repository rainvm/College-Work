\documentclass[12pt]{article}

\newcounter{results}
\newcounter{questions}

\def\neg{{\sim}}
\def\Z{\mathbb{Z}}
\def\N{\mathbb{N}}
\def\R{\mathbb{R}}
\def\Q{\mathbb{Q}}
\def\E{\mathbb{E}}
\def\qed{\(\blacksquare\)}
\newcommand{\result}[1]{\stepcounter{results}{\bfseries Result \arabic{results}}: #1}
\newcommand{\question}[1]{\stepcounter{questions}{\bf \arabic{questions}}: #1}
\newenvironment{proof}[2][Proof]{\result{#2}\begin{trivlist} 
    \item[\hskip \labelsep {\sc #1:}]}{\qed\end{trivlist}}
\usepackage{array}
\usepackage{amsmath}
\usepackage{amssymb}
\usepackage{mathtools}
\usepackage{textcomp}
\usepackage{gensymb}
\usepackage{graphicx}
\usepackage{float}
\usepackage{caption}
\usepackage{amsfonts}
\usepackage[margin=1in]{geometry}

\def\bar{\rule{\linewidth}{0.4pt}}

\title{Homework 5}
\author{Ryan Coyne}

\begin{document}

\maketitle

\section*{Question 1}
Prove the following theorem:
\begin{theorem}
    Let \(H_1\) and \(H_2\) be groups, and define
    \begin{equation*}
        H_1 \times \{e\} := \left\{ (h_1, e) | h_1\in H_1 \right\} \subset H_1 \times H_2
    \end{equation*}
    and
    \begin{equation*}
       \{e\} \times H_2 := \left\{ (e, h_2) | h_2\in H_2 \right\} \subset H_1 \times H_2.
    \end{equation*}
    Then
    \begin{enumerate}[label=\alph*)]
        \item \(H_1\times \left\{ e \right\}\) and \(\left\{ e \right\} \times H_2\) are normal subgroups of \(H_1 \times H_2\),
        \item \((H_1\times \left\{ e \right\}) \cap (\left\{ e \right\}\times H_2)\) = \(\left\{ (e,e) \right\}\), and
        \item \((H_1\times\left\{ e \right\})(\left\{ e \right\}\times H_2) = H_1 \times H_2\).
    \end{enumerate}
\end{theorem}
\bar
    \begin{enumerate}[label=\alph*)]
        \item \begin{proof}
            Let \((h_1, h_2)\in H_1\times H_2\) so \(h_1\in H_1\) and \(h_2 \in H_2\) and let \((x, e) \in H_1\times\{e\}\) so \(x\in H_1\). Now, 
            \begin{equation*}
                \begin{split}
                    (h_1, h_2)\circ(x, e)\circ(h_1, h_2)^{-1} &= (h_1, h_2)\circ(x, e)\circ(h_1^{-1}, h_2^{-1})\\
                    &= (h_1xh_1^{-1}, h_2eh_2^{-1})\\
                    &= (h_1xh_1^{-1}, h_2h_2^{-1})\\
                    &=(h_1xh_1^{-1}, e).
                \end{split}
            \end{equation*}
            Since, \(h_1, h_1^{-1}, x\in H_1\), it follows that \(h_1xh_1^{-1}\in H_1\). Thus, \((h_1, h_2)\circ(x, e)\circ(h_1, h_2)^{-1} = (h_1xh_1^{-1}, e)\in H_1\times H_2\) and therefore \(H_1\times \{e\}\) is normal. \\
            Now, let \(y\in H_1\). We have that
            \begin{equation*}
                \begin{split}
                    (h_1, h_2)\circ(e, y)\circ(h_1, h_2)^{-1} &= (h_1, h_2)\circ(e, y)\circ(h_1^{-1}, h_2^{-1})\\
                    &= (h_1eh_1^{-1}, h_2yh_2^{-1})\\
                    &= (h_1h_1^{-1}, h_2yh_2^{-1})\\
                    &=(e, h_2yh_2^{-1}).
                \end{split}
            \end{equation*}
            Since, \(h_2, h_2^{-1}, y\in H_1\), it follows that \(h_2yh_2^{-1}\in H_2\). Thus, \((h_1, h_2)\circ(y, e)\circ(h_1, h_2)^{-1} = (e, h_2yh_2^{-1})\in H_1\times H_2\) and therefore \(\{e\}\times H_2\) is normal. 
        \end{proof}
        \item \begin{proof}
            Let \(x \in (H_1\times \left\{ e \right\}) \cap (\left\{ e \right\}\times H_2)\). Now, \(x \in \left\{ e \right\}\times H_2\) and \(x \in H_1\times \left\{ e \right\}\). Thus, \((h_1, e) = x = (e, h_2)\) for some \(h_1\in H_1\) and \(h_2\in H_2\). Since \((h_1, e) = (e, h_2)\) it follows that \(h_1 = e\) and \(h_2 = e\). Thus \(x = (e, e)\) and so, \((H_1\times \left\{ e \right\}) \cap (\left\{ e \right\}\times H_2)\) = \(\left\{ (e,e) \right\}\).
        \end{proof}
        \item \begin{proof}
            By definition
            \begin{equation*}
                \begin{split}
                    (H_1\times\{ e \})(\{ e \}\times H_2) &=\{(h_1, e)(e, h_2)|h_1 \in H_1 \text{ and } h_2 \in H_2\}\\
                    &= \{(h_1e, eh_2)|h_1 \in H_1 \text{ and } h_2 \in H_2\}\\
                    & = \{(h_1, h_2)|h_1 \in H_1 \text{ and } h_2 \in H_2\}\\
                    &= H_1 \times H_2.
                \end{split}
            \end{equation*}
        \end{proof}
    \end{enumerate}
    
\section*{Question 2}
Prove the following theorem:
\begin{theorem}
    Let \(G\) be a group, and \(H\) a subgroup of \(G\). Then H is normal if and only if \(gH = Hg\) for all \(g \in G\).
\end{theorem}
\bar\\
\begin{proof}
    \\(\(\Leftarrow\)) Let \(h\in H\) and \(g\in G\). Now, \(gh = hg\) and so, \(ghg^{-1} = h \in H\). Therefore, \(H\) is normal.\\
    (\(\Rightarrow\)) Suppose H is normal. Then, \(gh_1g^{-1}=h_2\in H\) for some \(h_1\in H\) and \(g\in G\). It follows that \(gh_1 = h_2g\). Now, we define a function, \(f:H\rightarrow H\), as
    \begin{equation}
        f(h) = ghg^{-1}
    \end{equation}   
    and let \(x \in \ker(f)\). Then, \(gxg^{=1} = e\), and so \(x = g^{-1}g = e\). Therefore \(\ker(f) = \{e\} \) and \(f\) is injective. Since \(f\) is a function between two finite sets of the same size, it must also be surjective. Thus, \(f\) produces a different \(h_2 \in H\) for each \(h_1\in H\), and every \(h_2\) is produced by some \(h_1\). Therefore, \(gH = Hg\) for all \(g\in G\).
\end{proof}

\end{document}