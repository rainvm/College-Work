\documentclass[12pt]{article}

\newcounter{results}
\newcounter{questions}

\def\neg{{\sim}}
\def\Z{\mathbb{Z}}
\def\N{\mathbb{N}}
\def\R{\mathbb{R}}
\def\Q{\mathbb{Q}}
\def\E{\mathbb{E}}
\def\qed{\(\blacksquare\)}
\newcommand{\result}[1]{\stepcounter{results}{\bfseries Result \arabic{results}}: #1}
\newcommand{\question}[1]{\stepcounter{questions}{\bf \arabic{questions}}: #1}
\newenvironment{proof}[2][Proof]{\question{#2}\begin{trivlist} 
    \item[\hskip \labelsep {\sc #1:}]}{\qed\end{trivlist}}



\usepackage{upgreek}
\usepackage{sansmath}

\usepackage{pgfplots}
\usepackage{tikz}
\pgfplotsset{width=0.75\linewidth, compat = newest}

\usepackage[greek,english]{babel}

\usepackage{makeidx}
\usepackage{hyperref}
\usepackage{graphicx}
\usepackage{geometry}
%\usepackage[letterpaper, total={7.5in, 8in}]{geometry}

\usepackage{pseudocode}
\usepackage{calrsfs}

\usepackage{xifthen}% provides \isempty test
\usepackage{float}


\usepackage{amsmath}
\usepackage{amsfonts}
\usepackage{amsthm}
\usepackage{amssymb}

% \usepackage{enumitem}



\usepackage{caption}


\title{Theorems for Exam 1}
\author{Ryan Coyne}

\begin{document}

\maketitle

\setlength{\parindent}{0pt}

\begin{theorem}
    Let (\(G, \circ\)) be a group, and let \(g\in G\). Then
    \begin{equation*}
        g^{-1} \circ g = e.
    \end{equation*}
\end{theorem}

\begin{proof}
    We define \(x = g^{-1}\circ g\). Then we have
    \begin{equation*}
        \begin{split}
            x &= x\circ e\\
            & = x \circ x \circ x^{-1}\\
            &= g^{-1} \circ g \circ g^{-1} \circ g \circ x^{-1}\\
            &= g^{-1} \circ g \circ x^{-1}\\
            &= x \circ x^{-1}\\
            &= e
        \end{split}
    \end{equation*}
    Thus \(g^-1\circ g = e\).
\end{proof}

\begin{theorem}
    Let \((G, \circ)\) be a group and \(H\subseteq G\). Then \(H\) is a group if and only if:
    \begin{enumerate}[label=\alph*)]
        \item \(H\neq \emptyset\), and
        \item \(h_1 \circ h_2^{-1}\in H\) for all \(h_1, h_2\in H\).
    \end{enumerate}
\end{theorem}
\begin{proof}
    First, suppose that \(H\) is a subgroup. We get that \(e\in H\), so \(H\neq 0\). Thus, (a) holds. Now, let \(h_1, h_2\in H\), then we have that \(h_2^{-1}\in H\). Thus, \(h_1\circ h_2^{-1}\in H\). Hence (b) holds. 

    Now, suppose that (a) and (b) hold. From (a) we have that \(H\neq 0\), so there exists \(h\in H\). Thus from (b) we get that 
    \begin{equation*}
        e = h\circ h^{-1}\in H
    \end{equation*}
    and so (1) holds. 

    Let \(h\in H\). We have shown that \(e\in H\). So, from (b) we get
    \begin{equation*}
        h^{-1} = e \circ h^{-1} \in H.
    \end{equation*}
    So, (2) holds.
    
    Let \(h_1, h_2\in H\). We have shown that \(h_2^{-1}\in H\). Hence (b) gives that
    \begin{equation*}
        h_1\circ h_2= h_1\circ (h_2^{-1})^{-1} \in H.
    \end{equation*}
    Thus, (3) holds. Therefore \(H\) is a subgroup. 
\end{proof}

\begin{theorem}
    Let \((G, \circ)\) be a group, and \(H\) a subgroup of \(G\). Then for all \(g_1, g_2\in G\) there exists a bijection \(g_1H\rightarrow g_2H\). 
\end{theorem}
\begin{proof}
    Let \(x\in g_1H\). We define
    \begin{equation*}
        f(x) = g_2g_1^{-1}x.
    \end{equation*}
    As \(x = g1h\) for some \(h\in H\), we have that \(f(x) = g_2g_1^{-1}g_1h = g_2h\in g_2H\). Therefore \(f(x) \in g_2H\) and \(f\) is a function \(g_1H\rightarrow g_2H\). 

    We next claim that \(f\) is a bijection.\\ 
    inj) Let \(x_1, x_2\in g_1H\) such that \(f(x_1) = f(x_2)\). Then we have
    \begin{equation*}
        \begin{split}
            g_2g_2^{-1}x_1 = g_2g_1^{-1}x_2 &\Longrightarrow g_1^{-1}x_1 = g_1^{-1}x_2\\
            &\Longrightarrow x_1 = x_2. 
        \end{split}
    \end{equation*}
    Thus, \(f\) is injective.\\
    sur) Let \(y \in g_2H\). Then there exists \(h\in H\) such that \(y = g_2h\). We define
    \begin{equation*}
        x = g_1h\in g_1H.
    \end{equation*}
    We then have
    \begin{equation*}
        f(x) = g_2g_1^{-1}g_1h = g_2h = y.
    \end{equation*}
    Thus, \(f\) is bijective.
\end{proof}

\begin{theorem}
    Let \((G_1, \circ)\) and \((G_2, *)\) be groups, and \(f\) : \(G_1 \rightarrow G_2\) a group homomorphism. Then, \(\ker(f)\) is a subgroup of \(G_1\).
\end{theorem}

\begin{proof}
    As \(f(e_{G_1}) = e_{G_2}\), we have that \(e_{G_1}\in \ker(f)\), and so \(\ker(f)\neq \emptyset\). 
    
    Let \(g_1, g_2\in ker(f)\). Then,
    \begin{equation*}
        f(g_1) = e_{G_2} = f(g_2).
    \end{equation*}
    Therefore
    \begin{equation*}
        f(g_1\circ g_2^{-1}) = f(g_1)*f(g_2)^{-1} = e_{G_2} * e_{G_2}^{-1} = e_{G_2}.
    \end{equation*}
    Thus, \(g_1\circ g_2^{-1}\in \ker(f)\) and \(\ker(f)\) is a subgroup of \(G_1\).
\end{proof}

\begin{theorem}
    Let \((G_1, \circ)\) and \((G_2, *)\) be groups, and \(f\) : \(G_1 \rightarrow G_2\) a group homomorphism. Then, \(f\) is injective if an only if \(\ker(f) = \{e_{G_1}\}\).
\end{theorem}

\begin{proof}
    \(\Longrightarrow\) ) Assume \(f\) is injective, and let \(g\in \ker(f)\). THen \(f(g) = e_{G_2} = f(e_{G_1})\). As \(f\) is injective, we get that \(g=e_{G_1}\), and so \(\ker(f)= \{e_{G_1}\}\). 

    \(\Longleftarrow\) ) Assume \(\ker(f) = \{e_{G_1}\}\) and let \(g_1, g_2\in G_1\) such that \(f(g_1) = f(g_2)\). We then have
    \begin{equation*}
        \begin{split}
            &f(g_1) * f(g_2)^{-1} = e_{G_2}\\
            \Longrightarrow & f(g_1\circ g_2^{-1}) = e_{G_2}\\
            \Longrightarrow & g_1 \circ g_2^{-1}\in \ker(f)\\
            \Longrightarrow & g_1\circ g_2^{-1} = e_{G_1}\\
            \Longrightarrow & g_1 = g_2.
        \end{split}
    \end{equation*}
    Hence \(f\) is injective.
\end{proof}

\begin{theorem}
    Let \(G\) and \(H\) be groups, then \((G\times H, \circ)\) is a group where
    \begin{equation*}
        (g_1, h_1)\circ(g_2, h_2) := (g_1g_2, h_1h_2).
    \end{equation*}
\end{theorem}

\end{document}