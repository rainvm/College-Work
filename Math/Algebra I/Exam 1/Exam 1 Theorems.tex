\documentclass[12pt]{article}

\newcounter{results}
\newcounter{questions}

\def\neg{{\sim}}
\def\Z{\mathbb{Z}}
\def\N{\mathbb{N}}
\def\R{\mathbb{R}}
\def\Q{\mathbb{Q}}
\def\E{\mathbb{E}}
\def\qed{\(\blacksquare\)}
\newcommand{\result}[1]{\stepcounter{results}{\bfseries Result \arabic{results}}: #1}
\newcommand{\question}[1]{\stepcounter{questions}{\bf \arabic{questions}}: #1}
\newenvironment{proof}[2][Proof]{\question{#2}\begin{trivlist} 
    \item[\hskip \labelsep {\sc #1:}]}{\qed\end{trivlist}}



\usepackage{upgreek}
\usepackage{sansmath}

\usepackage{pgfplots}
\usepackage{tikz}
\pgfplotsset{width=0.75\linewidth, compat = newest}

\usepackage[greek,english]{babel}

\usepackage{makeidx}
\usepackage{hyperref}
\usepackage{graphicx}
\usepackage{geometry}
%\usepackage[letterpaper, total={7.5in, 8in}]{geometry}

\usepackage{pseudocode}
\usepackage{calrsfs}

\usepackage{xifthen}% provides \isempty test
\usepackage{float}


\usepackage{amsmath}
\usepackage{amsfonts}
\usepackage{amsthm}
\usepackage{amssymb}

% \usepackage{enumitem}



\usepackage{caption}


\title{Theorems for Exam 1}
\author{Ryan Coyne}

\begin{document}

\maketitle

\begin{theorem}
    Let (\(G, \circ\)) be a group, and let \(g\in G\). Then
    \begin{equation*}
        g^{-1} \circ g = e.
    \end{equation*}
\end{theorem}

\begin{proof}
    We define \(x = g^{-1}\circ g\). Then we have
    \begin{equation*}
        \begin{split}
            x &= x\circ e\\
            & = x \circ x \circ x^{-1}\\
            &= g^{-1} \circ g \circ g^{-1} \circ g \circ x^{-1}\\
            &= g^{-1} \circ g \circ x^{-1}\\
            &= x \circ x^{-1}\\
            &= e
        \end{split}
    \end{equation*}
    Thus \(g^-1\circ g = e\).
\end{proof}

\begin{theorem}
    Let \((G, \circ)\) be a group and \(H\subseteq G\). Then \(H\) is a group if and only if:
    \begin{enumerate}[label=\alph*)]
        \item \(H\neq \emptyset\), and
        \item \(h_1 \circ h_2^{-1}\in H\) for all \(h_1, h_2\in H\).
    \end{enumerate}
\end{theorem}
\begin{proof}
    First, suppose that \(H\) is a subgroup. We get that \(e\in H\), so \(H\neq 0\). Thus, (a) holds. Now, let \(h_1, h_2\in H\), then we have that \(h_2^{-1}\in H\). Thus, \(h_1\circ h_2^{-1}\in H\). Hence (b) holds. 

    Now, suppose that (a) and (b) hold. From (a) we have that \(H\neq 0\), so there exists \(h\in H\). Thus from (b) we get that 
    \begin{equation*}
        e = h\circ h^{-1}\in H
    \end{equation*}
    and so (1) holds. 

    Let \(h\in H\). We have shown that \(e\in H\). So, from (b) we get
    \begin{equation*}
        h^{-1} = e \circ h^{-1} \in H.
    \end{equation*}
    So, (2) holds.
    
    Let \(h_1, h_2\in H\). We have shown that \(h_2^{-1}\in H\). Hence (b) gives that
    \begin{equation*}
        h_1\circ h_2= h_1\circ (h_2^{-1})^{-1} \in H.
    \end{equation*}
    Thus, (3) holds. Therefore \(H\) is a subgroup. 
\end{proof}

\begin{theorem}
    Let \((G, \circ)\) be a group, and \(H\) a subgroup of \(G\). Then for all \(g_1, g_2\in G\)    
\end{theorem}

\end{document}