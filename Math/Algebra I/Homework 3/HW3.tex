\documentclass[12pt]{article}

\newcounter{results}
\newcounter{questions}

\def\neg{{\sim}}
\def\Z{\mathbb{Z}}
\def\N{\mathbb{N}}
\def\R{\mathbb{R}}
\def\Q{\mathbb{Q}}
\def\E{\mathbb{E}}
\def\qed{\(\blacksquare\)}
\newcommand{\result}[1]{\stepcounter{results}{\bfseries Result \arabic{results}}: #1}
\newcommand{\question}[1]{\stepcounter{questions}{\bf \arabic{questions}}: #1}
\newenvironment{proof}[2][Proof]{\question{#2}\begin{trivlist} 
    \item[\hskip \labelsep {\sc #1:}]}{\qed\end{trivlist}}



\usepackage{upgreek}
\usepackage{sansmath}

\usepackage{pgfplots}
\usepackage{tikz}
\pgfplotsset{width=0.75\linewidth, compat = newest}

\usepackage[greek,english]{babel}

\usepackage{makeidx}
\usepackage{hyperref}
\usepackage{graphicx}
\usepackage{geometry}
%\usepackage[letterpaper, total={7.5in, 8in}]{geometry}

\usepackage{pseudocode}
\usepackage{calrsfs}

\usepackage{xifthen}% provides \isempty test
\usepackage{float}


\usepackage{amsmath}
\usepackage{amsfonts}
\usepackage{amsthm}
\usepackage{amssymb}

% \usepackage{enumitem}



\usepackage{caption}


\def\g{\langle g \rangle}

\title{Homework 3}
\author{Ryan Coyne}

\begin{document}
    \maketitle

    \section*{Question 1}
    Let (\(G, \circ\)) be a group, \(g\in G\) and \(n\in \Z\). We define the notation
    \begin{equation*}
        \begin{split}
            g^0 &:= e\\
            g^n &:= \underbrace{g\circ g\circ\cdots\circ g}_{n\text{ times}}\\
            g^{n-1} &:= \underbrace{g^{-1}\circ g^{-1}\circ \cdots \circ g^{-1}}_{n\text{ times}}.
        \end{split}
    \end{equation*}
    \begin{enumerate}[label=\alph*)]
        \item Show that
        \begin{equation*}
            \langle g \rangle := \{g^n | n \in \Z\}
        \end{equation*}
        is a subgroup of \(G\).\\[12pt]
        \begin{proof}
            \ The set \(\langle g \rangle\) is non-empty because \(e = g^0 \in\g\). Next, let \(g^n, g^m\in \g\) for some \(m,n\in \Z\). Thus,
            \begin{equation*}
                \begin{split}
                    g^n \circ g^{-m} &= \underbrace{g\circ g\circ \cdots \circ g}_{n \text{ times}} \circ \underbrace{g^{-1} \circ g^{-1} \circ \cdots \circ g^{-1}}_{m \text{ times}}.
                \end{split}
            \end{equation*}
            If \(n>m\), then each \(g^{-1}\) is canceled by a \(g\). Thus,
            \begin{equation*}
                \begin{split}
                    g^n \circ g^{-m} &= \underbrace{g\circ g\circ \cdots \circ g}_{n-m \text{ times}}\\
                    &= g^{n-m}.
                \end{split}
            \end{equation*}
            Now, \(g^n \circ g^{-m}\in\g\) because \(n-m\in\Z\).\\
            If \(n<m\), then each \(g\) is cancelled by a \(g^{-1}\) and we have
            \begin{equation*}
                \begin{split}
                    g^n \circ g^{-m} &= \underbrace{g^{-1}\circ g^{-1}\circ \cdots \circ g^{-1}}_{m-n \text{ times}}\\
                    &= (g^{-1})^{m-n}\\
                    &= g^{n-m}
                \end{split}                
            \end{equation*}
            Now, \(g^n \circ g^{-m}\in\g\) because \(n-m\in\Z\).\\
            If \(n=m\), then each \(g\) is cancelled by a \(g^{-1}\) and vice-versa. Therefore, 
            \begin{equation*}
                \begin{split}
                    g^n \circ g^{-m} &= e.
                \end{split}
            \end{equation*}
            Now, \(g^n \circ g^{-m}\in\g\) because \(e\in\g\). Therefore, \(g^n \circ g^{-m}\in\g\) for all \(m, n\in\Z\).
        \end{proof}
        \item Consider the group \((S_5, 0)\). Determine the elements of \(\langle (12)(345) \rangle\).\\
        The set \(\langle (12)(345) \rangle = \{(), (12)(345), (354), (12), (345), (12)(543)\}\)
    \end{enumerate}

    \section*{Question 2}
    Consider the group \((S_4, \circ )\). Find one subgroup of \(S_4\) with:
    \begin{enumerate}[label=\alph*)]
        \item 2 elements: \(\{(), (12)\}\)
        \item 3 elements: \{(), (123), (132)\}
        \item 4 elements: \{(), (1234), (13)(24), (1432)\}
        \item 6 elements: \{(), (12), (23), (123), (132), (13)\}
        \item 8 elements: \{(), \}
        \item 12 elements: \{(), \}
    \end{enumerate}
    You do not need to justify that your answers are subgroups.
\end{document}