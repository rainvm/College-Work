\documentclass[12pt]{article}

\newcounter{results}
\newcounter{questions}

\def\neg{{\sim}}
\def\Z{\mathbb{Z}}
\def\N{\mathbb{N}}
\def\R{\mathbb{R}}
\def\Q{\mathbb{Q}}
\def\E{\mathbb{E}}
\def\qed{\(\blacksquare\)}
\newcommand{\result}[1]{\stepcounter{results}{\bfseries Result \arabic{results}}: #1}
\newcommand{\question}[1]{\stepcounter{questions}{\bf \arabic{questions}}: #1}
\newenvironment{proof}[2][Proof]{\question{#2}\begin{trivlist} 
    \item[\hskip \labelsep {\sc #1:}]}{\qed\end{trivlist}}



\usepackage{upgreek}
\usepackage{sansmath}

\usepackage{pgfplots}
\usepackage{tikz}
\pgfplotsset{width=0.75\linewidth, compat = newest}

\usepackage[greek,english]{babel}

\usepackage{makeidx}
\usepackage{hyperref}
\usepackage{graphicx}
\usepackage{geometry}
%\usepackage[letterpaper, total={7.5in, 8in}]{geometry}

\usepackage{pseudocode}
\usepackage{calrsfs}

\usepackage{xifthen}% provides \isempty test
\usepackage{float}


\usepackage{amsmath}
\usepackage{amsfonts}
\usepackage{amsthm}
\usepackage{amssymb}

% \usepackage{enumitem}



\usepackage{caption}


\title{Homework 4}
\author{Ryan Coyne}

\begin{document}

\maketitle

\section*{Question 1}
\begin{enumerate}[label=\alph*)]
    \item Prove the following theorem
    \begin{theorem}
        Let \((G, \circ)\) be a group, and \(H_1, H_2\subseteq G\) two subgroups of \(G\). Then \(H_1\cap H_2\) is a subgroup of \(G\).
    \end{theorem}
    \begin{proof}
        We begin by showing that \(H_1\cap H_2\) is non-empty. Since \(H_1\) is a subgroup of \(G\), \(e\in H_1\) and because \(H_2\) is also a subgroup of \(G\), \(e\in H_2\). It then follows that \(e\in H_1\cap H_2\). 
        
        Next we show that \(h_1\circ h_2^{-1}\in H_1\cap H_2\) for all \(h_1, h_2\in H_1\cap H_2\). Let \(h_1, h_2\in H_1 \cap H_2\). Now, \(h_1, h_2\in H_1\) and \(h_1, h_2\in H_2\). Because, \(H_1\) and \(H_2\) are subgroups of \(G\), we have that \(h_1\circ h_2^{-1}\in H_1\) and \(h_1\circ h_2^{-1}\in H_2\) and so \(h_1\circ h_2^{-1}\in H_1\cap H_2\). Therefore \(H_1\cap H_2\) is a subgroup of \(G\).
    \end{proof}
    \item Give an example of a group \((G, \circ)\), and two subgroups \(H_1, H_2\subseteq G\) such that \(H_1 \cup H_2 \) is {\textbf not} as subgroup of \(G\).

    One such group is \(S_3\), with the subgroups \(H_1 = \{(), (12)\}\) and \(H_2 = \{(), (23)\}\). Now, \(H_1\cup H_2 = \{(), (12), (23)\}\) is not a subgroup becuase \((12)\circ (23) = (13) \not\in H_1\cup H_2\) .
\end{enumerate}

\section*{Question 2}
Let \(N\in\N_{\geq 1}\), and consider the group \(GL(N), \cdot\). We define
\begin{equation*}
    O(N) := \left\{ M\in GL(N)| M^T\cdot M = I_N \right\}
\end{equation*}
where \(M^T\) denotes the matrix transpose of \(M\). Show that \(O(N)\) is a subgroup of \(GL(N)\).

\begin{proof}
    We begin by showing that the identity element, \(I_N\), is in \(O(N)\). The identity matrix is always its own transpose, that is to say, \(I_N = I_N^T\). Thus, 
    \begin{equation*}
        \begin{split}
            I_N^T \cdot I_N &= I_N\cdot I_N \\ &= I_N.
        \end{split}
    \end{equation*}
    The identity element is therefore in \(O(N)\). 

    Next, we show that for all \(M\in O(N)\) there exists \(M^{-1}\in O(N)\). Let \(M\in O(N)\), and thus, \(M^T\cdot M = I_N\). It follows that \(M^T = M^{-1}\). Now,
    \begin{equation*}
        \begin{split}
            (M^T)^T\cdot M^T &= M\cdot M^T\\
            &= M \cdot M^{-1}\\
            &= I_N.
        \end{split}
    \end{equation*}
    Thus, \(M^T\in O(N)\). Since \(M^T = M^{-1}\), we have that \(M^{-1}\in O(N)\) for all \(M\in O(N)\).

    Lastly, we show that \(M_1 \cdot M_2 \in O(N)\) for all \(M_1, M_2\in O(N)\). Let \(M_1, M_2 \in O(N)\). Now, because \(M_1^T = M_1^{-1}\) and \(M_2^T = M_2^{-1}\), \(M_1\) and \(M_2\) are orthonormal matrices, and all orthonomal matrices in \(GL(N)\) are in \(O(N)\) and so \(M_1 \cdot M_2\) is also orthonormal. Now, \((M_1 \cdot M_2)^T \cdot (M_1 \cdot M_2) = I_N\), and thus \(M_1 \cdot M_2\in O(N)\).
    
    Therefore, \(O(N)\) satisfies all of the conditions for being a subgroup of \(GL(N)\).
\end{proof}
\end{document}