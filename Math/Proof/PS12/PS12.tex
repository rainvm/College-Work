\documentclass[12pt]{article}

\usepackage{array}
\usepackage{amsmath}
\usepackage{amssymb}
\usepackage{mathtools}
\usepackage{textcomp}
\usepackage{gensymb}
\usepackage{graphicx}
\usepackage{float}
\usepackage{caption}
\usepackage{amsfonts}
\usepackage[margin=1in]{geometry}
\newenvironment{proof}[2][Proof]{\result{#2}\begin{trivlist} 
    \item[\hskip \labelsep {\sc #1:}]}{\qed\end{trivlist}}
\newcounter{results}
\newcounter{questions}

\def\neg{{\sim}}
\def\Z{\mathbb{Z}}
\def\N{\mathbb{N}}
\def\R{\mathbb{R}}
\def\Q{\mathbb{Q}}
\def\E{\mathbb{E}}
\def\qed{\(\blacksquare\)}
\newcommand{\result}[1]{\stepcounter{results}{\bfseries Result \arabic{results}}: #1}
\newcommand{\question}[1]{\stepcounter{questions}{\bf \arabic{questions}}: #1}

\newcommand{\mc}[1]{\mathcal{#1}}

\allowdisplaybreaks

\begin{document}
\title{Problem Set 12}
\author{Ryan Coyne}
\date{}
\maketitle

\begin{enumerate}
    \item \begin{enumerate}
        \item The relation \(R_1\) is a function since there are no elements in the domain, \(\R\), that cannot be mapped to an element in the codomain, \(\R\), and \(y=4x-3\) produces only one \(y\) value for each \(x\) value.
        \item The relation \(R_2\) is not a function because each element \(a\in A_2\), where \(a>0\), maps to two elements of the codomain. For example, \((4,0)\) and \((4,-4)\) are both elements of \(R_2\).
        \item The relation \(R_3\) is not a function because each element \(a\in A_3\) to two elements, \(b_1, b_2 \in \R\). In particular \(b_1\) and \(b_2\) are related by \(b_2 = b_1 + 4\). As an example, \((0,-2)\) and \((0,2)\) are both in \(R_3\).
    \end{enumerate}
    \item \begin{enumerate}
        \item \(f(n) = n\)
        \item \(f(n) = n + 1\)
        \item \(f(n) = \begin{cases}
            1 ,& x = 1\\
            n-1, & x > 1
        \end{cases}\)
        \item \(f(n) = 1\)
    \end{enumerate}
    \item The relation, \(\mc F\), is a set such that, \(\mc F = \{(2,7), (4,1), (6, 4), (6, 9)\}\). Thus, this relation is not a function because \(\mc F(6) = 4, 9\).
    \item \begin{enumerate}
        \item Let \(x\in A\). Now, \((f\circ f\circ f)(x) = f(f(f(x)))\). Thus, \begin{equation*}
            \begin{split}
                f(f(f(x))) &= 1 - \frac{1}{1 - \frac{1}{1 - \frac{1}{x}}}\\
                &= 1- \frac{1}{\frac{1-\frac{1}{x}-1}{1-\frac{1}{x}}}\\
                &= 1 - \frac{1}{\frac{\frac{1}{x}}{\frac{1}{x}-1}}\\
                &= 1 - \frac{\frac{1}{x}-1}{\frac{1}{x}}\\
                &= 1 - (1-x)\\
                &=x
            \end{split}
        \end{equation*}
        Therefore \(f\circ f\circ f\) is the identity function.
        \item Since, \(f \circ f \circ f\) is the identity function, it follows that \(f\circ f^{-1} = f \circ f \circ f\). Thus, \begin{equation*}
            \begin{split}
                f^{-1}(x) &= f \circ f(x) = 1 - \frac{1}{1-\frac{1}{x}}\\
                &= 1 - \frac{1}{\frac{x-1}{x}}\\
                &= 1 - \frac{x}{x-1}\\
                &= \frac{x-1-x}{x-1}\\
                &= \frac{1}{1-x}
            \end{split}
        \end{equation*}
    \end{enumerate}
    \item \begin{enumerate}
        \item The function, \(F\), is not one-to-one because \(F(2)=F(4)=0\). In fact, \(F(a)=0\) for any \(a\in(\N\cup\{0\})\cap\E\), where \(\E\) is the set of even integers.
        \item I conjecture that it is onto, but I cannot find a proof of it. 
    \end{enumerate}
    \item Let \(T,S\) be sets. Now, \(T-S = \{a\in T| a\not\in S\} = \{a\in T| a\not \in T\cap S\}\). Thus, \(T-S = T - T\cap S\). By the same process, it can be shown that \(S - T = S - T\cap S\). Assuming the hypothesis that \(|T-S| = |S-T|\), it follows that ,\(|T - T\cap S| = |S - T\cap S|\). Now since, \(T\cap S \subset T\) and \(T\cap S\subset S\), it is the case that \(|T|-|T\cap S| = |S| - |T\cap S|\). Finally, by algebraic manipulation, \(|T|=|S|\). 
\end{enumerate}

\end{document}