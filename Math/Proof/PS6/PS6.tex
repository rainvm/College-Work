\documentclass[12pt]{article}
\usepackage{array}
\usepackage{amsmath}
\usepackage{amssymb}
\usepackage{mathtools}
\usepackage{gensymb}
\usepackage{graphicx}
\usepackage{float}
\usepackage{caption}
\usepackage{amsfonts}

\renewcommand*{\neg}{{\sim}}
\newcommand{\Z}{\mathbb{Z}}

\allowdisplaybreaks

\begin{document}
    \title{Problem Set 6}
    \author{Ryan Coyne}
    \maketitle

    \begin{enumerate}
        \item Let \(x,y,z\in \mathbb{Z}\). Prove: If exactly two of \(x,y,z\) are even, then \(3x+5y+7z\) is odd.\\ Case 1: Let \(x, y\) be even and let \(z\) be odd. Then, \(x=2k\), \(y=2l\), and \(z = 2m+1\), for some \(k, l, m\in\Z\). Now, \begin{equation*}
            \begin{split}
                3x+5y+7z &= 6k + 10l + 14m+7\\
                &=6k+10l+14m+6+1\\
                &= 2(3k+5l+7m+3)+1
            \end{split}
        \end{equation*}
        which is odd, by definition.\\
        Case 2: Let \(x,z\) be even and let \(y\) be odd. Then, \(x=2k\), \(y=2l+1\), and \(z = 2m\), for some \(k, l, m\in\Z\). Now, \begin{equation*}
            \begin{split}
                3x+5y+7z &= 6k + 10l + 5 + 14m\\
                &=6k+10l+14m+4+1\\
                &= 2(3k+5l+7m+2)+1
            \end{split}
        \end{equation*}
        which is odd, by definition.\\
        Case 3: Let \(y,z\) be odd and let \(x\) be odd. Then, \(x=2k+1\), \(y=2l\), and \(z = 2m\), for some \(k, l, m\in\Z\). Now, \begin{equation*}
            \begin{split}
                3x+5y+7z &= 6k + 3 + 10l + 14m\\
                &=6k+10l+14m+2+1\\
                &= 2(3k+5l+7m+1)+1
            \end{split}
        \end{equation*}
        which is odd, by definition.\pagebreak
        \item Let \(a,b\in\Z\). Prove: If \(ab=4\), then \((a-b)^3-9(a-b)=0\).\\ Case 1: Let \(a=1\) and \(b=4\). Then, \begin{equation*}
            \begin{split}
                (a-b)^3-9(a-b)&=(1-4)^3-9(1-4)\\
                &=-3^3-9\cdot-3\\
                &=-27+27\\
                &=0.
            \end{split}
        \end{equation*}
        Case 2: Let \(a=1\) and \(b=4\). Then, \begin{equation*}
            \begin{split}
                (a-b)^3-9(a-b)&=(-1+4)^3-9(-1+4)\\
                &=3^3-9\cdot3\\
                &=27-27\\
                &=0.
            \end{split}
        \end{equation*}
        Case 3: Let \(a=2\) and \(b=2\). Then, \begin{equation*}
            \begin{split}
                (a-b)^3-9(a-b)&=(2-2)^3-9(2-2)\\
                &=0^3-9\cdot 0\\
                &=0.
            \end{split}
        \end{equation*}
        Case 4: Let \(a=-2\) and \(b=-2\). Then, \begin{equation*}
            \begin{split}
                (a-b)^3-9(a-b)&=(-2+2)^3-9(-2+2)\\
                &=0^3-9\cdot 0\\
                &=0.
            \end{split}
        \end{equation*}
        Therefore, by exhaustion, \((a-b)^3-9(a-b)=0\).
        \item Let \(a\in\Z\). Prove: If \(3 \mid 2a\), then \(3\mid a\).\\ By Result 4.8 from the textbook, if \(3\mid cd\), then \(3\mid c\) or \(3\mid d\), for some \(c, d\in\Z\). Since \(3\mid 2a\) and \(3\nmid 2\), then it must be the case that \(3\mid a\).
        \item Let \(x,y\in\Z\). Prove: If 3 divides neither \(x\) or \(y\), then \(3\mid (x^2-y^2)\).\\ Since \((x^2-y^2)\) can be factored into \((x+y)(x-y)\), \(3\mid(x^2-y^2)\) exactly when \(3\mid(x+y)\) or \(3\mid(x-y)\). Proceeding by cases according to the remainder of 3 divided by \(x\) and the remainder of 3 divided by \(y\). \\  (i) Let \(x=3k+1\), and \(y=3l+1\) for some \(k,l\in\Z\). Then,
        \begin{equation*}
            \begin{split}
                x-y&=3k+1-3l-1\\
                &=3k-3l\\
                &=3(k-l),
            \end{split}
        \end{equation*}
        which is divisible by 3.\\ (ii) Let \(x=3k+2\), and \(y=3l+2\) for some \(k,l\in\Z\). Then,
        \begin{equation*}
            
        \end{equation*}
    \end{enumerate}
\end{document}