\documentclass[12pt]{article}
\usepackage{array}
\usepackage{amsmath}
\usepackage{mathtools}
\usepackage{gensymb}
\usepackage{graphicx}
\usepackage{float}
\usepackage{caption}
\usepackage{amsfonts}

\renewcommand*{\neg}{{\sim}}

\allowdisplaybreaks

\begin{document}
    \title{Problem Set 5}
    \author{Ryan Coyne}
    \maketitle
    \begin{enumerate}
        \item \begin{enumerate}
            \item \(\forall z \in U, \exists x \in S, \exists y \in T,~ z=x+y \)
            \item \(\forall x \in S, \forall y \in T, \exists z \in S,~ z = xy\)
            \item \(\forall x\in S, \exists y \in T,~ y>x\)
        \end{enumerate}
        \item\begin{enumerate}
            \item True
            \item True
            \item False
            \item True
            \item True
            \item False
            \item True
        \end{enumerate}
        \item Option (d) implies that \((\neg P(x)) \implies Q(x)\) is false for some \(x\in \mathbb{Q}\). The others do not.
        \item \begin{enumerate}
            \item For all circles, \(C_1\), in \(\mathcal{A}\) there is at least one circle, \(C_2\), in \(\mathcal{B}\), such that \(C_1\) and \(C_2\) have exactly two points in common.
            \item \(\exists C_1 \in \mathcal{A}, \forall C_2 \in \mathcal{B}, \neg P(C_1, C_2)\).
            \item There exists a circle, \(C_1\), in \(\mathcal{A}\), such that there is no circle, \(C_2\), in \(\mathcal{B}\), for which \(C_1\) and \(C_2\) share exactly two points.
            \item The statement in (a) is true. The statements in (b) and (c) are false.
        \end{enumerate}
        \item \begin{enumerate}
            \item This is true.  If two lines are perpendicular, all angles between them are \(90\degree\). Since \(\ell_1\) and \(\ell_2\) are perpendicular to \(\ell_3\), the angles between \(\ell_1\) and \(\ell_3\) and the angles between \(\ell_2\) and \(\ell_3\) will all be \(90\degree\). Thus, the corresponding angles will be congruent, and by the corresponding angles theorem, \(\ell_1\) and \(\ell_2\) are parallel.
            \item This is true. Two lines are parallel if they never intersect. Any third line that does intersect \(\ell_1\) is parallel to \(\ell_1\), and if that line also doesn't intersect \(\ell_2\) it is parallel to \(\ell_2\). Since \(\ell_1\) and \(\ell_2\) are parallel to the same line, they are parallel by the parallel transitive theorem.
            \item This is true by the corresponding angles theorem.
            \item This is false. Parallel lines do not cover the entire space that they exist in.
        \end{enumerate}
        \item \(\forall a,b,c \in S, a + b + c = 0 \implies abc < 0\), where \(S = \{x |x = 2k + 1, k\in\mathbb{Z}\}\).\\
        Let \(a = 2k+1\), \(b=2l + 1\), \(c = 2m + 1\), where \(k, l, m \in \mathbb{Z}\). \\
        \(a + b + c= 2k+1 + 2l+1 + 2m+1\)\\
        \(~~~~~~~~~~~~\, =2(k+l+m + 1) + 1\)\\
        Therefore \(\forall a, b, c\in S\), \(a + b + c \neq 0\). Thus, the implication is true for all \(a, b, c \in S\) since the premise is always false. 
        \item Prove: \(\forall k \in \mathbb{Z}, \exists x \in \mathbb{Z}, x=2k \implies \exists l \in \mathbb{Z}, 7x-3 = 2l + 1\).\\
        \(7x-3 = 7(2k)-3\)\\
        \(~~~~~~~~~\,=2(7k)-3\)\\
        \(~~~~~~~~~\,=2(7k-2) + 1\)\\
        Let \(l = 7k-2\).\\
        Therefore, \(7x-3=2l+1\).\\
        Prove: \(\forall l \in \mathbb{Z}, \exists x \in \mathbb{Z}, 7x-3 = 2l + 1 \implies \exists k \in \mathbb{Z}, x = 2k \). \\
        \(7x-3=2l-1\)\\
        \(7x = 2l+2\)\\
        \(x = \frac{2}{7}(l+1)\)\\
        Since \(x \in \mathbb{Z}\), it follows that \(\frac{l+1}{7} \in \mathbb{Z}\). \\
        Let \(k=(l+1)/7\).\\
        Therefore \(x = 2k\).\\\\\\
        \item Prove: \(\forall k\in \mathbb{Z}, \exists x \in \mathbb{Z}, 3x-1=2k \implies \exists l \in \mathbb{Z}, 5x+2 = 2l+1\).\\
        Suppose x is even.\\
        \(x= 2m\), \(m\in\mathbb{Z}\).\\
        \(6m-1 = 2k\).\\
        \(m = \frac{1}{3}k - \frac{1}{6}\#\).
        Therefore, \(x\) must be odd.\\
        \(x = 2m+1\)\\
        \(5(2m+1) + 2 = 2l+1\)\\
        \(10m + 7 = 2l+1\)\\
        \(2(5m + 3) + 1 = 2l+1\)\\
        Therefore, if \(3x-1\) is even, \(5x + 2\) must be odd.\\
        Prove: \(\forall l \in \mathbb{Z}, \exists x \in \mathbb{Z}, 5x + 2 = 2l + 1 \implies \exists k\in \mathbb{Z}, 3x-1=2k\).\\
        Suppose \(x\) is even.\\
        \(x = 2m\), \(m \in \mathbb{Z}\)\\
        \(5(2m)+2=2l+1\)\\
        \(10m + 2 = 2l + 1\)\\
        \(m = \frac{1}{5}l + \frac{1}{10}\#\). Therefore, \(x\) is odd.\\
        \(x = 2m + 1\)\\
        \(3(2m+1) -1 = 2k\)\\
        \(6m + 2 = 2k\)\\
        \(2(3m+1)= 2k\)\\
        Therefore, if \(5x+2\) is odd, \(3x-1\) must bee even.
    \end{enumerate}
\end{document}