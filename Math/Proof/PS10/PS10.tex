\documentclass[12pt]{article}
\usepackage{array}
\usepackage{amsmath}
\usepackage{amssymb}
\usepackage{mathtools}
\usepackage{textcomp}
\usepackage{gensymb}
\usepackage{graphicx}
\usepackage{float}
\usepackage{caption}
\usepackage{amsfonts}
\usepackage[margin=1in]{geometry}

\newenvironment{proof}[2][Proof]{\result{#2}\begin{trivlist} 
    \item[\hskip \labelsep {\sc #1:}]}{\qed\end{trivlist}}
\newcounter{results}
\newcounter{questions}

\def\neg{{\sim}}
\def\Z{\mathbb{Z}}
\def\N{\mathbb{N}}
\def\R{\mathbb{R}}
\def\Q{\mathbb{Q}}
\def\E{\mathbb{E}}
\def\qed{\(\blacksquare\)}
\newcommand{\result}[1]{\stepcounter{results}{\bfseries Result \arabic{results}}: #1}
\newcommand{\question}[1]{\stepcounter{questions}{\bf \arabic{questions}}: #1}

\allowdisplaybreaks

\begin{document}

    \title{Problem Set 9}
    \author{Ryan Coyne}
    \maketitle

    \noindent\begin{proof}{Find a formula for \begin{equation*}
        1+4+7+ \cdots (3n-2)
    \end{equation*}
    for positive integers then verify your formula by mathematical induction.\\
    The formula is \begin{equation*}
        1+4+7+ \cdots (3n-2) = \frac{n(3n-1)}{2}
    \end{equation*}
    for all positive integers, \(n\).}
    We will prove by induction that 
    \begin{equation*}
        1+4+7+ \cdots (3n-2) = \frac{n(3n-1)}{2}
    \end{equation*}
    for all positive integers \(n\). 
    Base Step: Let \(n=1\). Now, \(3n-2=3-2=1\), and so the sum is equal to one. Now, consider the right side \begin{equation*}
        \begin{split}
            \frac{1\cdot(3\cdot 1-1)}{2} &= \frac{3-1}{2}\\
            &=\frac{2}{2}\\
            &=1.
        \end{split}
    \end{equation*}
    Thus, the formula is valid for \(n=1\).\\
    We show that,
    \begin{equation*}
        1+4+7+ \cdots (3(k+1)-2) = \frac{(k+1)(3(k+1)-1)}{2}
    \end{equation*}
    Now, assume that \(1+4+7+ \cdots (3k-2) = \frac{k(3k-1)}{2}\) is valid for all \(k\in \N\). Add \(3(k+1)-2\) to both sides,
    \begin{equation*}
        \begin{split}
            1+4+7+ \cdots (3k-2) + (3k+1) &= \frac{k(3k-1)}{2} + 3k+1.
        \end{split}
    \end{equation*}
    Next, write the right side with a common denominator and rearrange the numerator into our desired form,
    \begin{equation*}
        \begin{split}
            1+4+7+ \cdots (3k-2) + (3k+1) &= \frac{k(3k-1)+ 6k+2}{2}\\
            &=\frac{3k^2-k + 6k+2}{2}\\
            &=\frac{3k^2+5k+2}{2}\\
            &=\frac{(k+1)(3k+2)}{2}\\
            &=\frac{(k+1)(3(k+1)-1)}{2}.
        \end{split}
    \end{equation*}
    Thus, by the principle of mathematical induction, \(1+4+7+ \cdots (3n-2) = \frac{n(3n-1)}{2}\).
    \end{proof}
    \begin{proof}{Prove the following inequality for every positive integer n:\begin{equation*}
        2!\cdot4!\cdot6!\cdots(2n)!\geq((n+1)!)^n.
    \end{equation*}}
        We proceed by induction. Since \((2\cdot1)!=(2!)^1\), the statement is true when \(n=1\). Assume that
        \begin{equation}
            2!\cdot4!\cdot6!\cdots(2k)!\geq((k+1)!)^k
        \end{equation}
        for some integer, \(k\). We show,
        \begin{equation*}
            \begin{split}
                2!\cdot4!\cdot6!\cdots(2k)!\cdot(2(k+1))&\geq((k+2)!)^{k+1}.
            \end{split}
        \end{equation*}
        Now, multiply either side of equation (1) by \((2(k+1))!\),
        \begin{equation*}
            2!\cdot4!\cdot6!\cdots(2k)!\cdot(2(k+1))!\geq((k+1)!)^k\cdot(2(k+1))!.
        \end{equation*}
        Now, we show that \(((k+1)!)^k\cdot(2(k+1))!\geq((k+2)!)^{k+1}\) directly. Divide both sides by \(((k+1)!)^k\),
        \begin{equation*}
            \begin{split}
                \frac{((k+1)!)^k\cdot(2(k+1))!}{((k+1)!)^k}&\geq\frac{((k+2)!)^{k+1}}{((k+1)!)^k}\\
                (2k+2)! &\geq \frac{((k+1)!)^{k+1}(k+2)^{k+1}}{((k+1)!)^{k}}\\
                &\geq (k+1)!(k+2)^{k+1}.
            \end{split}
        \end{equation*}
        Now, divide by \((k+1)!\),
        \begin{equation*}
            \begin{split}
                \frac{(2k+2)!}{(k+1)!}&\geq \frac{(k+1)!(k+2)^{k+1}}{(k+1)!}\\
                \underbrace{(k+2)\cdot(k+3)\cdots(2k+1)\cdot(2k+2)}_{k+1\text{ factors}} & \geq (k+2)^{k+1}.
            \end{split}
        \end{equation*}
        Note that both sides involve the multiplication of integers \(k+1\) times. However, all of the factors on the right are \(k+2\), and exactly one of the factors on the left is \(k+2\). Additionally, \(k\) of the factors on the left are greater than \(k+2\). Thus, it is true that \(((k+1)!)^k\cdot(2(k+1))!\geq((k+2)!)^{k+1}\), and by induction.
    \end{proof}
    \begin{proof}{Prove that for every real number \(x>-1\) and every positive integer \(n\), \begin{equation*}
        (1+x)^n\geq 1+nx
    \end{equation*}}
    
    \end{proof}
    \begin{proof}{Prove that \(81|(10^{n+1}-9n-10) \) for every positive integer \(n\).}
        Base Case: Let \(n=1\). Now, \(10^2-9 - 10 = 81\). Thus, \(81|(10^{n+1}-9n-10)\) when \(n=1\).\\
        Induction Step: Now, assume that \(81|(10^{k+1}-9k-10)\). Next, consider \(10^{k+2}-9(k+1)-10 = 10^{k+2} - 9k -19\). Now, subtract \(10(10^{k+1} -9k-10)\) from both sides
        \begin{equation*}
            \begin{split}
                10^{k+2}-9(k+1)-10 - 10(10^{k+1} -9k-10) &= 10^{k+2} - 9k -19 - 10(10^{k+1} -9k-10)\\
                &=10^{k+2} - 9k -19 - 10^{k+2} + 90k + 100\\
                &= 81(k+1).\\
            \end{split}
        \end{equation*}
        Then, add \(10(10^{k+1} -9k-10)\) to both sides,
        \begin{equation*}
            \begin{split}
                10^{k+2}-9(k+1)-10 &= 81(k+1) + 10(10^{k+1} -9k-10).
            \end{split}
        \end{equation*}
        Since, \(81|(10^{k+1}-9k-10)\), then \(10^{k+1}-9k-10=81j\), for some \(j\in\N\). Thus, 
        \begin{equation*}
            \begin{split}
                10^{k+2}-9(k+1)-10 &= 81(k+1) + 10(81j)\\
                &=81(k+1) + 81(10j)\\
                &=81(k+1+10j).
            \end{split}
        \end{equation*}
        Thus, \(81|(10^{k+2}-9(k+1)-10)\). Therefore, \(81|(10^{n+1}-9n-10) \) for all \(n\in\N\).
    \end{proof}
    \begin{proof}{A sequence \(\{a_n\}\) is defined recursively by
        \begin{equation*}
            a_1 = 1,\ a_2=2;\ a_n=a_{n-1} + 2a_{n-2},
        \end{equation*}
        for \(n\geq3\). Conjecture a formula for \(a_n\) and verify that your conjecture is correct.}
        We prove by induction that for the sequence defined above, \(a_n=2^{n-1}\). \\
        Since, \(a_1=2^0=1\), the formula holds for \(n=1\). Assume for an arbitrary \(k\) that \(a_i=2^{i-1}\) for every integer \(i\) with \(1\leq i \leq k\). We show that \(a_{k+1} = 2^k\). If \(k=1\), then \(a_{k+1} = a_2=2^1=2\). Since, \(a_2=2\), it follows that \(a_{k+1} = 2^k\) for \(k=1\). Now, we may assume that \(k\geq2\). Since \(k+1\geq3\), it follows that
        \begin{equation*}
            \begin{split}
                a_{k+1} &= a_{k} + 2a_{k-1}\\
                & = 2^{k-1} + 2\cdot2^{k-2}\\
                & = 2^{k-1} + 2^{k-1}\\
                & = 2\cdot 2^{k-1}\\
                &= 2^{k},
            \end{split}
        \end{equation*}
        which is the desired result. By the strong principle of mathematical induction, \(a_n=2^{n-1}\), for all \(n\in\N\).
    \end{proof}
    \begin{proof}{Consider the sequence of Fibonacci numbers \(\{F_n\}\), where
        \begin{equation*}
            F_1 = 1,\ F_2 =1,\ F_n=F_{n-1} + F_{n-2},
        \end{equation*}
        for \(n\geq3\).\\
        \textbf{(a)} Prove \(2|F_n\) if and only if \(3|n\).}
        Since, \(3|n\), it is clear that \(n=3m\) for some \(m\in N\). In the case where \(m=1\), we have
        \begin{equation*}
            F_n=F_3=F_2+F_1=1+1=2
        \end{equation*}
    \end{proof}
\end{document}