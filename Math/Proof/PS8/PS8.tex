\documentclass[12pt]{article}
\usepackage{array}
\usepackage{amsmath}
\usepackage{amssymb}
\usepackage{mathtools}
\usepackage{gensymb}
\usepackage{graphicx}
\usepackage{float}
\usepackage{caption}
\usepackage{amsfonts}
\usepackage[margin=1in]{geometry}

\renewcommand*{\neg}{{\sim}}
\newcommand{\Z}{\mathbb{Z}}
\newcommand{\N}{\mathbb{N}}
\newcommand{\R}{\mathbb{R}}
\newcommand{\Q}{\mathbb{Q}}
\newcommand{\qed}{\(\blacksquare\)}

\allowdisplaybreaks

\begin{document}
    \title{Problem Set 7}
    \author{Ryan Coyne}
    \maketitle

    \begin{enumerate}
        \item Prove: the product of an irrational number and a nonzero rational number is irrational.\\ Proof: Suppose that \(xy=z\), for some \(x,z\in\Q\) and \(y\in\R-\Q\). By definition, \(x=\frac{a}{b}\) and \(z=\frac{a'}{b'}\), for some \(a,a',b,b'\in\Z\), with \(a,a',b,b'\neq0\). Then, \(\frac{a}{b}y=\frac{a'}{b'}\), and so, \(y=\frac{a'b}{b'a}\). Now, \(a'b\) and \(b'a\), are integers, and so \(y\) must be rational by definition. This contradicts the initial assumption that \(y\) is irrational, and thus, the product of an irrational and a nonzero rational number, cannot be rational. The product must therefore be rational. \qed
        \item  
    \end{enumerate}
\end{document}