\documentclass[12pt]{article}
\usepackage{array}
\usepackage{amsmath}
\usepackage{amssymb}
\usepackage{mathtools}
\usepackage{gensymb}
\usepackage{graphicx}
\usepackage{float}
\usepackage{caption}
\usepackage{amsfonts}
\usepackage[margin=1in]{geometry}

\renewcommand*{\neg}{{\sim}}
\newcommand{\Z}{\mathbb{Z}}
\newcommand{\N}{\mathbb{N}}
\newcommand{\R}{\mathbb{R}}
\newcommand{\Q}{\mathbb{Q}}
\newcommand{\qed}{\(\blacksquare\)}

\allowdisplaybreaks

\begin{document}
    \title{Problem Set 8}
    \author{Ryan Coyne}
    \maketitle

    \begin{enumerate}
        \item \textbf{Prove:} the product of an irrational number and a nonzero rational number is irrational.\\ \textbf{Proof:} Assume, to the contrary, that \(xy=z\), for some \(x,z\in\Q\) and \(y\in\R-\Q\). By definition, \(x=\frac{a}{b}\) and \(z=\frac{a'}{b'}\), for some \(a,a',b,b'\in\Z\), with \(a,a',b,b'\neq0\). Then, \(\frac{a}{b}y=\frac{a'}{b'}\), and so, \(y=\frac{a'b}{b'a}\). Now, \(a'b\) and \(b'a\), are integers, and so \(y\) must be rational by definition. This contradicts the initial assumption that \(y\) is irrational. Thus, the product of an irrational and a nonzero rational number cannot be rational and must, therefore, be rational. \qed
        \item \textbf{Prove:}  \(\sqrt{2}+\sqrt{3}\) is an irrational number.\\ \textbf{Proof:} Assume, to the contrary, that \(\sqrt{2}+\sqrt{3}\) is rational, then \((\sqrt{2}+\sqrt{3})^2\) must also be rational. Now, \((\sqrt{2}+\sqrt{3})^2=5+2\sqrt{6}\). If \(\sqrt{6}\) is irrational, then \(5+2\sqrt{6}\) is also irrational. Suppose that \(\sqrt{6}\) is rational. Let \(\sqrt{6}=\frac{a}{b}\), for some \(a,b\in\Z\), with \(a\) and \(b\) having no common factors. Then, \(6=\frac{a^2}{b^2}\), and so \(b^2=\frac{a^2}{6}\). Since, \(b\) is an integer, \(b^2\) is also an integer. Thus, \(6|a^2\) and \(6|a\). By definition, \(a=6k\) for some \(k\in\Z\). Now, we substitute \(6k\) for a and simplify,\begin{equation*}
            \begin{split}
                b^2&=\frac{(6k)^2}{6}\\
                &=6k^2.
            \end{split}
        \end{equation*}
        So, \(b\) must have \(6\) as a factor. Since \(a\) and \(b\) both have \(6\) as a factor, this contradicts our assumption that they have no common factors. Now, \(\sqrt{6}\) must be irrational, and therefore \(\sqrt{2}+\sqrt{3}\) is also irrational. \qed
        \item \textbf{Prove:} there do not exist three distinct real numbers \(a\), \(b\), and \(c\) such that all of the numbers \(a+b+c\), \(ab\), \(ac\), \(bc\), and \(abc\) are equal.\\ \textbf{Proof:} Assume, to the contrary, \(a+b+c=ab=ac=bc=abc\), with \(a,b,c\) being distinct real numbers. Now, by substituting \(ab\), \(bc\), and \(ac\) into \(abc\) we can obtain, \(abc=ac^2=ab^2=bc^2\). Then, \(ac^2=ab^2\) can only be true when \(b=c\), \(b=-c\), or \(a=0\). Since \(b=c\) is disallowed, consider the cases \(a=0\) and \(b=-c\). \\ \textbf{Case 1}: Let \(a=0\). Now, \(abc=0\), and so \(abc=bc=0\). Without loss of generality, let \(b=0\). Then, \(b=a=0\), which is disallowed.\\ \textbf{Case 2}: Let \(b=-c\). Now, \(bc^2=ac^2\) is true if \(a=b\) or \(c=0\). The former, \(a=b\), is trivially disallowed, and if \(c=0\) and \(b=-c\), then \(b=0\) and \(b=c\) which is also disallowed. \\Therefore, there are no possible distinct values for \(a\), \(b\), and \(c\) in the real numbers. \qed
        \item Let \(a,b,c,d\) be real numbers. \textbf{Prove:} at most four of the numbers \(ab\), \(ac\), \(ad\), \(bc\), \(bd\), and \(cd\) are negative. \\ \textbf{Proof:} We will consider the possible cases for \(a,b,c,d\) being negative. All possible products of pairs of \(a\), \(b\), \(c\), and \(d\) are represented so there is no qualitative difference between them, and we will examine a representative case for each number of negatives in \(a,b,c,d\). \\
        \textbf{Case 1:} Let \(a,b,c,d>0\). In this case, none of the products are negative.\\ 
        \textbf{Case 2:} Let \(a<0\) and \(b,c,d>0\). Now, \(ab,ac,ad<0\) and \(bc,bd,cd>0\). Three of these are negative.\\ 
        \textbf{Case 3:} Let \(a,b<0\) and \(c,d>0\). Now, \(ac,ad, bc, bd<0\) and \(ab, cd>0\). Four are negative.\\
        \textbf{Case 4:} Let \(a,b,c<0\) and \(d>0\). Now \(ad, cd, bd<0\) and \(ab, ac, bc>0\). Three are negative.\\
        \textbf{Case 5:} Let \(a,b,c,d<0\). Now, \(ab,ac,ad,bc,bd,cd<>0\). Zero are negative.\\
        All possibilities have been exhausted and therefore, at most four of the numbers \(ab\), \(ac\), \(ad\), \(bc\), \(bd\), and \(cd\) are negative, as in the case where \(a,b<0\) and \(c,d>0\). \qed
    \end{enumerate}
\end{document}