\documentclass[12pt]{article}
\usepackage{array}
\usepackage{amsmath}
\usepackage{amssymb}
\usepackage{mathtools}
\usepackage{gensymb}
\usepackage{graphicx}
\usepackage{float}
\usepackage{caption}
\usepackage{amsfonts}
\usepackage[margin=1in]{geometry}

\renewcommand*{\neg}{{\sim}}
\newcommand{\Z}{\mathbb{Z}}
\newcommand{\N}{\mathbb{N}}
\newcommand{\R}{\mathbb{R}}
\newcommand{\Q}{\mathbb{Q}}
\newcommand{\qed}{\(\blacksquare\)}

\allowdisplaybreaks

\begin{document}
    \title{Problem Set 8}
    \author{Ryan Coyne}
    \maketitle

    \begin{enumerate}
        \item \textbf{Prove:} the product of an irrational number and a nonzero rational number is irrational.\\ \textbf{Proof:} Assume, to the contrary, that \(xy=z\), for some \(x,z\in\Q\) and \(y\in\R-\Q\). By definition, \(x=\frac{a}{b}\) and \(z=\frac{a'}{b'}\), for some \(a,a',b,b'\in\Z\), with \(a,a',b,b'\neq0\). Then, \(\frac{a}{b}y=\frac{a'}{b'}\), and so, \(y=\frac{a'b}{b'a}\). Now, \(a'b\) and \(b'a\), are integers, and so \(y\) must be rational by definition. This contradicts the initial assumption that \(y\) is irrational. Thus, the product of an irrational and a nonzero rational number cannot be rational and must, therefore, be rational. \qed
        \item \textbf{Prove:}  \(\sqrt{2}+\sqrt{3}\) is an irrational number.\\ \textbf{Proof:} Assume, to the contrary, that \(\sqrt{2}+\sqrt{3}\) is rational, then \((\sqrt{2}+\sqrt{3})^2\) must also be rational. Now, \((\sqrt{2}+\sqrt{3})^2=5+2\sqrt{6}\). If \(\sqrt{6}\) is irrational, then \(5+2\sqrt{6}\) is also irrational. Suppose that \(\sqrt{6}\) is rational. Let \(\sqrt{6}=\frac{a}{b}\), for some \(a,b\in\Z\), with \(a\) and \(b\) being coprime. Then, \(6=\frac{a^2}{b^2}\), and so \(b^2=\frac{a^2}{6}\). Since, \(b\) is an integer, \(b^2\) is also an integer. Thus, \(6|a^2\). It follows that \(6|a\) since \(\sqrt{6}\not\in\Z\). By definition, \(a=\) Now, return to the equation \(6=\frac{a^2}{b^2}\). 
        \item \textbf{Prove:} there do not exist three distinct real numbers \(a\), \(b\), and \(c\) such that all of the numbers \(a+b+c\), \(ab\), \(ac\), \(bc\), and \(abc\) are equal.\\ \textbf{Proof:} Assume, to the contrary, \(a+b+c=ab=ac=bc=abc\), with \(a,b,c\) being distinct real numbers. Now, by substituting \(ab\), \(bc\), and \(ac\) into \(abc\) we can obtain, \(abc=ac^2=ab^2=bc^2\). Then, \(ac^2=ab^2\) can only be true when \(b=c\), \(b=-c\), or \(a=0\). Since \(b=c\) is disallowed, consider the cases \(a=0\) and \(b=-c\). \\ \textbf{Case 1}: Let \(a=0\). Now, \(abc=0\), and so \(abc=bc=0\). Without loss of generality, let \(b=0\). Then, \(b=a=0\), which is disallowed.\\ \textbf{Case 2}: Let \(b=-c\). Now, \(bc^2=ac^2\) is true if \(a=b\) or \(c=0\). The former, \(a=b\), is trivially disallowed, and if \(c=0\) and \(b=-c\), then \(b=0\) and \(b=c\) which is also disallowed. \\Therefore, there are no possible distinct values for \(a\), \(b\), and \(c\) in the real numbers. \qed
        \item Let \(a,b,c,d\) be real numbers. \textbf{Prove:} at most four of the numbers \(ab\), \(ac\), \(ad\), \(bc\), \(bd\), and \(cd\) are negative. \\ \textbf{Proof:} Assume, to the contrary, that five of the numbers \(ab\), \(ac\), \(ad\), \(bc\), \(bd\), and \(cd\) are negative. Let \(ab\), \(ac\), \(ad\), \(bc\), and \(bd\) be negative. Now, since \(ab\), \(ac\), and \(ad\)
    \end{enumerate}
\end{document}