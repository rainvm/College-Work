\documentclass[12pt]{article}
\usepackage{array}
\usepackage{amsmath}
\usepackage{mathtools}
\usepackage[letterpaper, total={7.5in, 8in}]{geometry}

\allowdisplaybreaks

\begin{document}
    \title{The Density of Wood and an Unknown Metal}
    \author{Ryan Coyne}
    \maketitle
    \section{Abstract}
        The density of a block of wood and a cylinder made from an unknown metal were determined
        from measurements of the dimensions and the mass. The dimensions of the wood block were
        measured using a ruler and the dimensions of the metal cylinder were measured using vernier
        calipers. The density of the wood was measured to be \(0.6129 \pm 0.0071\) \( \text{g}/\text{\text{cm}}^3\) and the
        density of the metal cylinder was measured to be \(2.7344 \pm 0.0278\) \(\text{g}/\text{\text{cm}}^3\).
    \section{Introduction}
        In this lab the density of two materials was calculated from the Mass of an object and certain
        lengths of that object. Density is the mass per unit volume of a material and is commonly
        expressed in units of kilograms per cubic meter or grams per cubic centimeter. 
    \section{Procedure}
        The length, width, and height of a rectangular prism made of wood were measured using a ruler
        with markings at each millimeter. The height and diameter of a cylinder made of an unknown
        metal were measured using vernier calipers with markings at each millimeter. The mass of 
        both objects was measured using a triple beam balance. The beam with the largest weight has 
        notches indicating hundreds of grams, the beam with the middle weight has notches 
        indicating tens of grams, and the beam with the smallest weight has markings indicating 
        tenths of grams.
    \section{Data}
        \begin{center}
            \begin{tabular} {>{\centering\arraybackslash}p{0.07\textwidth}|>{\centering\arraybackslash}p{0.1\textwidth}>{\centering\arraybackslash}p{0.1\textwidth}>{\centering\arraybackslash}p{0.1\textwidth}>{\centering\arraybackslash}p{0.1\textwidth}>{\centering\arraybackslash}p{0.1\textwidth}|}
                \multicolumn{5}{c}{Table 1. Wooden Block}\\ 
                Trial & $M(\text{g})$ & $L(\text{cm})$ & $W(\text{cm})$ & $H(\text{cm})$\\ 
                \hline
                1 & 73.72 & 6.19 & 6.02 & 3.25\\
                2 & 73.85 & 6.21 & 6.01 & 3.27\\
                3 & 73.80 & 6.22 & 5.99 & 3.21\\
                4 & 73.73 & 6.19 & 6.01 & 3.19\\
                \hline
                $\overline{x}$ & 73.775 & 6.203 & 6.008 & 3.230\\
                \(\sigma\) & 0.061 & 0.015 & 0.013 & 0.037\\
            \end{tabular}\\[12pt]

            \begin{tabular}{>{\centering\arraybackslash}p{0.05\textwidth}|>{\centering\arraybackslash}p{0.17\textwidth}>{\centering\arraybackslash}p{0.17\textwidth}>{\centering\arraybackslash}p{0.17\textwidth}}
                \multicolumn{4}{c}{Table 2. Metal Cylinder}\\
                Trial & $M(\text{g})$ & $H(\text{cm})$ & $D(\text{cm})$\\
                \hline
                1 & 65.81 & 5.01 & 2.47\\
                2 & 65.83 & 5.01 & 2.49\\
                3 & 65.85 & 5.01 & 2.46\\
                4 & 65.88 & 5.02 & 2.47\\
                \hline
                $\overline{x}$ & 65.843 & 5.0150 & 2.473\\
                $\sigma$ & 0.030 & 0.0058 & 0.013
            \end{tabular}\\[12pt]
            \begin{tabular}{c|c c c}
                \multicolumn{3}{c}{Table 3. Results}\\
                & Wood & Metal\\
                \hline
                $\rho$($\text{g}/\text{cm}^3$) & 0.6129 & 2.734\\
                $\sigma_{\rho}$ & 0.0071 & 0.028\\
            \end{tabular}\\[12pt]

        \end{center}

    \section{Calculations}
    \begin{align*}
        \overline{m}_w &= \frac{73.72\text{g}+73.85\text{g}+73.80\text{g}+73.73\text{g}}{4}\\
        &=73.775\text{g}\\
        \sigma_{mw}^2 &= \frac{(73.72\text{g}-73.775\text{g})^2+(73.85\text{g}-73.775\text{g})^2+(73.80\text{g}-73.775\text{g})^2+(73.73\text{g}-73.775\text{g})^2}{3}\\
        &=\frac{(-0.055\text{g})^2+(0.075\text{g})^2+(0.025\text{g})^2+(-0.045\text{g})^2}{3}\\
        &=\frac{0.0113\text{g}^2}{3}\\
        &=0.00376667\text{g}^2\\
        \sigma_{mw} &= \sqrt{0.00376667\text{g}^2}\\
        &=0.061\text{\text{g}}\\
        \frac{\sigma_{mw}}{\overline{m}_w} & = \frac{73.775\text{\text{g}}}{0.061\text{\text{g}}}\\
        &=0.083\%\\
        \rho_w &= \frac{73.775 \text{g}}{(6.203\text{cm})(6.008\text{cm})(3.230\text{cm})}\\
        &= 0.6129 \text{g}/\text{\text{cm}}^3\\
        \rho_{wM} &= \frac{73.775 \text{g}+0.061\text{g}}{(6.203\text{cm})(6.008\text{cm})(3.230\text{cm})}\\
        &= 0.6134\text{g}/\text{\text{cm}}^3\\
        \rho_{wL} &= \frac{73.775 \text{g}}{(6.203\text{cm}+0.015\text{cm})(6.008\text{cm})(3.230\text{cm})}\\
        &= 0.6115\text{g}/\text{\text{cm}}^3\\
        \rho_{wW} &= \frac{73.775\text{g}}{(6.203\text{cm})(6.008\text{cm}+0.013\text{cm})(3.230\text{cm})}\\
        &= 0.6117\text{g}/\text{\text{cm}}^3\\
        \rho_{wH} &= \frac{73.775\text{g}}{(6.203\text{cm})(6.008\text{cm})(3.230\text{cm}+0.037\text{cm})}\\
        &= 0.6061\text{g}/\text{\text{cm}}^3\\
        \sigma_{\rho w}^2 &= (0.6115\text{g}/\text{\text{cm}}^3-0.6129\text{g}/\text{\text{cm}}^3)^2 + (0.6117\text{g}/\text{\text{cm}}^3-0.6129\text{g}/\text{\text{cm}}^3)^2 + (0.6061\text{g}/\text{\text{cm}}^3-0.6129\text{g}/\text{\text{cm}}^3)^2\\
        &= 0.000051\text{g}^2/\text{\text{cm}}^6\\
        \sigma_{\rho w} & = \sqrt{0.000051}\\
        &= 0.0071\text{g}/\text{\text{cm}}^3\\
        \rho_{m} &= \frac{65.843\text{g}}{\pi 5.0150\text{cm} (\frac{2.473\text{cm}}{2})^2}\\
        &= 2.734\text{g}/\text{\text{cm}}^3\\
        \sigma_{\rho m }^2 &= (2.731\text{g}/\text{\text{cm}}^3-2.734\text{g}/\text{\text{cm}}^3)^2 + (2.706\text{g}/\text{\text{cm}}^3-2.734\text{g}/\text{\text{cm}}^3)^2\\
        &= 0.000773\text{g}^2/\text{\text{cm}}^6\\
        \sigma_{\rho m } & = \sqrt{0.000773\text{g}^2/\text{\text{cm}}^6}\\
        &= 0.028\text{g}/\text{\text{cm}}^3
    \end{align*}

    \section{Conclusion}
        Overall this experiment succeeded in determining the density of these two materials. The 
        published density of aluminum is within two standard deviations of the measured density 
        of the cylinder in this experiment. The measured density of the wood block is well within 
        the range of published values for types of 
        wood, but there is such a wide range of densities for wood and the densities of many species 
        of wood overlap so it is not possible to say what species of wood was measured without 
        further investigation. Random error arose in this experiment from limitations in the precision 
        of the instruments used for measurement. These results could be improved with more precise 
        measurement instruments.
\end{document}