\documentclass[12pt]{article}
\usepackage{array}
\usepackage{amsmath}
\usepackage{mathtools}
\usepackage{textcomp, gensymb}
\usepackage{graphicx}
\usepackage{float}
\usepackage{caption}
\usepackage{enumitem}

\allowdisplaybreaks

\setlist[enumerate]{label=(\alph*)}

\begin{document}
    \title{Activity 2}
    \author{Ryan Coyne}
    \maketitle

    \section*{Problem 1}

    \begin{enumerate}
        \item This is exact because a gram is defined as 1/1000 kilograms.
        \item This is approximate because the census is not responded to by every person in the United States, and we can't be confident that every response is perfectly accurate.
        \item This is exact because the inch is defined this way.
        \item This is exact because %TODO
        \item This is exact because it is counted.
        \item This is approximate because it's not possible to place the lines at exactly a 1mm distance every time.
    \end{enumerate}

    \section*{Problem 2}
    \begin{enumerate}
        \item \(h\pm\delta h = (5.03\pm0.04)\) m
        \item \(t\pm\delta t = (20\pm1)\) s
        \item \(q\pm\delta q = (-3.2\pm0.3)\times10^{-19}\) C
        \item \(\lambda\pm\delta\lambda = (5.6\pm0.7)\times10^{-7}\) m
        \item \(p\pm\delta p = (3.27\pm0.04)\times10^3\) g \(\cdot\) cm/s 
    \end{enumerate}
    
    \section*{Problem 3}
    \begin{enumerate}
        \item \(\overline{V} = \frac{1}{8}(25.8\text{ ml} + 26.2\text{ ml} + 26\text{ ml} + 26.5\text{ ml} + 25.8\text{ ml} + 26.1\text{ ml} + 25.8\text{ ml} + 26.3\text{ ml})\\~~~\,= 26.0625\text{ ml}\)
        \item \(\delta V_{\text{type A}} = \sqrt{\frac{1}{8(8-1)}\sum_{i=1}^{8}(V_i-26.0625\text{ ml})^2}\\~~~~\quad\quad~=0.09\) ml
        \item \(\delta V = \sqrt{\delta V^2_\text{Type A} + \delta V^2_\text{Type B}}\\~~~~~=\sqrt{0.09^2 + 0.1^2}\\~~~~~=0.13\) ml
        \item \(V\pm\delta V = (26.06\pm0.13)\) ml
    \end{enumerate}

    \section*{Problem 4}
    \begin{enumerate}
        \item \begin{enumerate}
            \item[(i)] It is incorrect to believe all of the digits of the quoted speed because the distance and time measurements were only precise to two significant figures and so the result of the calculation is also only precise to two significant figures.
            \item[(ii)] The speed should be reported as \(19\) m/s.
        \end{enumerate}
        \item 
    \end{enumerate}

    \section*{Problem 5}
    I would estimate that the Type B uncertainty is equal to half of the distance of the smallest marked interval. In the case of the rule being shown, I would estimate it to be 0.5 mm.

    \section*{Problem 6}
    According to the user manual, the uncertainty of AC volt measurements up to 4 V is 
    \section*{Problem 7}

\end{document}