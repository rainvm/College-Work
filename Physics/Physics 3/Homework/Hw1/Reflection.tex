\documentclass[12pt]{article}
\usepackage{array}
\usepackage{amsmath}
\usepackage{mathtools}
\usepackage{textcomp, gensymb}
\usepackage{graphicx}
\usepackage{float}
\usepackage{caption}
\usepackage{enumitem}

\allowdisplaybreaks

\begin{document}
    \title{Homework 1 Reflection}
    \author{Ryan Coyne}
    \maketitle
    
    \section{Questions}
    \begin{enumerate}
        \item 
    \end{enumerate}

    \section{Problems}
    \begin{enumerate}
        \item  The only real difference between your solution and my solution is skipped the step where you wrote \(\Delta t = \gamma\Delta t'\). 
        \item I used the same process it just wrote it out more explicitly. I don't really know how to improve that.
        \item Instead of calculating \(\gamma\) I calculated \(\frac{1}{\gamma}\) from the begining. I think it is better to do it this way.
        \item I didn't calculate \(\gamma\) separately first. I'm happy with the way I did this.
        \item I used a sufficiently powerful calculator instead of an approximation. For this problem, I wrote everything except the final answer down correctly. I got the wrong answer because I forgot the square root when I typed it into my calculator. 
        \item I did this the same way as you. No changes.
    \end{enumerate}

    \section{Challengers}
    \begin{enumerate}
        \item You found \(t'\) in terms of \(x'\) whereas I didn't bother and just left \(t'\) in terms of \(x\) after inverting, and then solved for \(x\) after the substitution. I prefer the way that I did it because it involved fewer steps.
        \item 
    \end{enumerate}
\end{document}