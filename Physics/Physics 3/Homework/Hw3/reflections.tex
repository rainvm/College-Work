\documentclass[12pt]{article}
\usepackage{array}
\usepackage{amsmath}
\usepackage{mathtools}
\usepackage{textcomp, gensymb}
\usepackage{graphicx}
\usepackage{float}
\usepackage{caption}
\usepackage{enumitem}

\allowdisplaybreaks

\begin{document}
    \title{Homework 3 Reflection}
    \author{Ryan Coyne}
    \maketitle
    
    \section{Questions}
    \begin{enumerate}
        \item I'm not entirely convinced that storing information necessarily stores energy. I could accept that it depends on the way that the information is stored because otherwise there are ways of storing information which would seem to anthropromophize the laws of physics. Storing energy in the batter on the other hand clearly should make the laptop slightly heavier.
        \item I went a different way because I had already done problem 7 and in doing so had derived an equation which would show that the speed of a massless particle must be \(c\).
        \item I just got this one totally correct.
    \end{enumerate}

    \section{Problems}
    \begin{enumerate}
        \setcounter{enumi}{3}
        \item \begin{enumerate}
            \item I did this one exactly like your solutions.
            \item This was totally correct.
        \end{enumerate}
        \item I had a hard time understanding exactly what this question wanted me to do, and after looking at the solutions I think I am still not understanding.
        \item \begin{enumerate}
            \item Just \(E=mc^2\). Simple calculuation.
            \item I used the formula \(E^2 = p^2c^2 + m_0^2c^4\) instead of the simpler \(E=\gamma m_pc^2\).
            \item I used \(m = \gamma m_0\) instead of \(E/c^2\) which I think was simpler since I already had a value for \(\gamma\).
            \item This one is identical to mine.
        \end{enumerate}
        \item \begin{enumerate}
            \item This solution is the same as mine.
            \item I didn't find \(\gamma\) first, but this is otherwise identical.
        \end{enumerate}
        \item \begin{enumerate}
            \item I found \(p^2\) and then took the square root. Effectively the same as your solution.
            \item I just used the \(p\) from part a instead of finding it again like you.
        \end{enumerate}
        \item \begin{enumerate}
            \item I used the formula I derived in problem 8 because it was right in front of me and so there was a lower cognitive load by using that.
            \item My solution is identical to yours.
            \item My solution is identical to yours.
        \end{enumerate}
        \item \begin{enumerate}
            \item The only difference is that I considered it as a rate vs time which I like better, honestly.
            \item This is part of why I like my answer for the previous part better. It is a clearer progression from the previous part to this one.
        \end{enumerate}
        \item \begin{enumerate}
            \item I didn't start by finding \(E\), I just went right to \(\gamma\).
            \item This one was simple since it was just classical kinetic energy.
            \item I think \(7\%\) is on the edge of mattering. If you are trying to do precise measurements, then that sort of error would matter, but if that's all people had experienced, they probably wouldn't notice. I didn't consider that they would be manually adjusted though which is interesting.
        \end{enumerate}
        \item I just wrote a lot less, but it's the same idea.
        \item I did the same thing, and jsut didn't write as much.
    \end{enumerate}

    \section{Challengers}
    \begin{enumerate}
        \setcounter{enumi}{13}
        \item I missed the \(-1\) that you have in equation (22), but I got quite close.
        \item I thought that \(v_max\) was the same as in the previous homework but now I see that he doesn't quite get there in half the time it takes to get to alpha centauri. 
    \end{enumerate}
\end{document}