\documentclass[12pt]{article}




\usepackage{upgreek}
\usepackage{sansmath}

\usepackage{pgfplots}
\usepackage{tikz}
\pgfplotsset{width=0.75\linewidth, compat = newest}

\usepackage[greek,english]{babel}

\usepackage{makeidx}
\usepackage{hyperref}
\usepackage{graphicx}
\usepackage{geometry}
%\usepackage[letterpaper, total={7.5in, 8in}]{geometry}

\usepackage{pseudocode}
\usepackage{calrsfs}

\usepackage{xifthen}% provides \isempty test
\usepackage{float}


\usepackage{amsmath}
\usepackage{amsfonts}
\usepackage{amsthm}
\usepackage{amssymb}

% \usepackage{enumitem}



\usepackage{caption}

\newenvironment{proof}[2][Proof]{\question{#2}\begin{trivlist} 
    \item[\hskip \labelsep {\sc #1:}]}{\qed\end{trivlist}}
\newcounter{results}
\newcounter{questions}

\def\neg{{\sim}}
\def\Z{\mathbb{Z}}
\def\N{\mathbb{N}}
\def\R{\mathbb{R}}
\def\Q{\mathbb{Q}}
\def\E{\mathbb{E}}
\def\qed{\(\blacksquare\)}
\newcommand{\result}[1]{\stepcounter{results}{\bfseries Result \arabic{results}}: #1}
\newcommand{\question}[1]{\stepcounter{questions}{\bf \arabic{questions}}: #1}

\begin{document}
    \title{Notes on "The Calabi-Yau Landscape"}
    \author{Ryan Coyne}
    \date{}
    \maketitle
    \section{Prologue}
    \begin{itemize}
        \item \(\Sigma\) - Surface
        \item \(g(\Sigma)\) - Genus of the surface, i.e. the number of holes.
        \item \(\chi(\Sigma) = 2 - 2g(\Sigma)\) - Euler characteristic
        \begin{itemize}
            \item Signed alternating sum of the number of independent objects in each dimension. For example, a cube drawn on \(S^2\) has 8 vertices, 12 edges and 6 faces, thus \(\chi(\Sigma) = 8 - 12 + 6 = 2\)
            \item Betti number, \(b_i\), counts the number of independent algebraic cycles in dimension \(i\).
            \item \(\chi(\Sigma) = \sum_{i=0}^2b_i\ ,\quad b_i = \mathrm{rk}(H_i(\Sigma))\)
        \end{itemize}
        \item 
    \end{itemize}

\end{document}