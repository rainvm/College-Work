\documentclass[12pt]{article}
\usepackage{array}
\usepackage{amsmath}
\usepackage{mathtools}
\usepackage{gensymb}
\usepackage{graphicx}
\usepackage{float}
\usepackage{caption}
\usepackage{setspace}


\allowdisplaybreaks

\begin{document}

    \title{Qualitative Observations of Charged Materials}
    \author{Ryan Coyne, Ben Eid, Erin Snook}
    \maketitle

    \section{Abstract}
        In this lab, we observed the effect that electric charge has on objects. Namely, that there is an attractive or repulsive force between charged objects, and that there are two types of charges that an object can have. 
    \section{Introduction}
        All of the matter that we experience daily is made up of atoms. These atoms have three primary constituents, protons, neutrons, and electrons. The proton has one unit of positive charge, the neutron has no charge, and the electron has one unit of negative charge. The protons and neutrons are bound together to form the nucleus of the atom and the electrons can be thought of as orbiting the nucleus. Before the atom or the particles of which it is comprised were discovered, there were known methods to cause electric charge to accumulate on objects.  

        Objects have a neutral electric charge by default. This is because the atoms that they are composed of typically contain an equal number of positively charged protons and negatively charged electrons, which cancel each other out and result in zero net charge. When, for example, plastic and wool are rubbed together, some of the electrons from one material are transferred to the other. This causes the material that gains electrons to accumulate a net negative charge and the material that loses electrons to gain a net positive charge. Plastic and wool maintain this charge because they are not effective conductors of electricity.

        When an object with a negative electric charge nears one with a positive electric charge, there is an attractive force between them. On the other hand, when two negatively charged objects or two positively charged objects are near one another, there is a repulsive force between them. 
    \section{Procedure}
    \subsection{Part I}
        In the first part, we used plastic tubes, glass rods, and acrylic tubes. When charging the plastic tubes we rubbed them with wool and when charging the glass rods and acrylic tubes we rubbed them with silk. For each test, we placed one or more rods or tubes horrizontally on a swivel and then held another to observe any potential forces between the objects.

        To begin with, we placed one undisturbed plastic tube on the swivel and brought another close to it. The tubes had no apparent effect on each other. Next, we rubbed both with wool and brought them close together. This time, the tube on the swivel moved away from the held tube which is indicative of a repulsive force between them. The same occurred when the experiment was repeated with two glass tubes rubbed with silk. When a glass rod that had been rubbed with silk was brought near a plastic tube that had been rubbed with wool, they were attracted. When observing the acceleration of the rod on the swivel, it is clear that the force between the objects is greater when they are closer together because the acceleration is greater than when they are farther apart.

        Next, these experiments were repeated after rubbing more or less vigorously. We observed that more vigorous rubbing and rubbing for a longer period of time tended to cause a greater force between the objects. This did not necessarily seem to be proportional to the vigorousness or time spent, and there appeared to be some upper limit. When a charged rod was held over small pieces of paper, the paper would rise and stick to the rod. This happened for both the plastic and glass rods but did not happen for any neutral rod.

        When a charged plastic tube and a charged glass rod were placed on swivels an appropriate distance apart and a neutral object was brought near them, they were both attracted to it. When the material that a rod was charged with was held close to the charged rod on a swivel there was an attractive force between them. When the wool that was used to charge the plastic was brought close to the charged glass rod, they repel and when the silk used to charge the glass was brought near the charged plastic, they also repel. More types of plastics and glass were charged and none could be found that would attract the charged original rods.

        From these experiments there seem to be two types of charge which can exert a force at a distance. When two objects have the same type of charge, there is a repulsive force between them and when they have different charges there is an attractive force. Neutral objects are also attracted to either type of charge, but not as strongly. The force increases with the amount of charge and decreases with the distance between the objects.

        \subsection{Part II}
        We charged a plastic rod by rubbing it with wool and touched a metal sphere with the rubbed area. When the sphere was held above small pieces of paper they were attracted to the sphere, indicating the sphere acquired a charge from the plastic. Next, we charged a plastic rod and then touched the charged area. The pieces of paper did not react after touching the charged area. We transferred charge from the plastic to the sphere again and after touching the sphere, it was no longer charged.

        Next, we placed a plastic rod between two spheres and touched one with a charged plastic rod. The sphere that was touched directly acquired a charge but the other did not. When this was repeated with a metal rod connecting the spheres, they both acquired a charge. These experiments indicate that plastic and metal can both acquire a charge, but the charge can move through the metal much more easily than the plastic, and charge is transferred by physical contact.
    \section{Conclusion}
        There are two types of charge. Like charges attract and opposite charges repel. This force between charges can act at a distance but acts more strongly at a shorter distance. More charge also causes a stronger force. When two objects are rubbed together to create charges, the charge in each is opposite from the other. Charge can be transferred by direct physical contact or through materials. Some materials such as metals are better at transferring charge within the material than others such as plastic.
\end{document}